Pushdown analysis improves upon finite-state analysis in precision and performance since it can rule out spurious paths caused by improperly matched calls and returns.
%
Despite this, we have not seen total widespread adoption of these techniques for analyzing higher-order languages.
%
A possible explanation is that the known techniques are technically burdened and difficult to understand or extend.
%
Control structure of the programming language gets pulled into the model of computation, which makes extensions to non-pushdown control structures, such as first-class control operators, non-trivial.
%%

%%
We show a derivational approach to abstract interpretation that yields transparently sound static analyses that can precisely match calls and returns when applied to well-known abstract machines.
%
We show that the abstraction techniques in the literature have an intuitive operational meaning, even in the concrete setting, that build off well-known programming techniques such as memoization.
%
This approach allows us to derive the existing, more technically involved analyses, and a novel pushdown analysis for delimited, composable control.
