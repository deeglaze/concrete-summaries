\documentclass[article]{llncs}

% preprint only
\pagestyle{plain}
\usepackage{amsmath,stmaryrd,listings,mathpartir,amssymb,graphicx}

\usepackage[usenames,dvipsnames]{color}

\usepackage{pfsteps} % for appendix

\newcommand{\Scribtexttt}[1]{{\texttt{#1}}}
\newcommand{\SColorize}[2]{\color{#1}{#2}}
\newcommand{\inColor}[2]{{\Scribtexttt{\SColorize{#1}{#2}}}}
\definecolor{PaleBlue}{rgb}{0.90,0.90,1.0}
\newcommand{\rackett}[1]{\inColor{black}{#1}}
\newcommand{\todo}[1]{\textbf{TODO:} #1}
\newcommand{\iftr}[1]{#1}

\newcommand{\eg}{\textit{e.g.}}
\newcommand{\ie}{\textit{i.e.}}

\newcommand{\nat}{{\mathbb N}}
% Metavariables from defined spaces
\newcommand{\mexpr}{e}
\newcommand{\mexpri}[1]{e_{#1}}
\newcommand{\mvar}{x}
\newcommand{\mval}{v}
\newcommand{\mvalpre}{v_\mathit{pre}}
\newcommand{\mvalpost}{v_\mathit{post}}
\newcommand{\mtag}{v_{\mathit{t}}}
\newcommand{\mhandler}{v_\mathit{h}}
\newcommand{\maddr}{a}
\newcommand{\menv}{\rho}
\newcommand{\mstore}{\sigma}
\newcommand{\mkont}{\kappa}
\newcommand{\mskont}{\hat{\kappa}}
\newcommand{\mmkontacc}{C_\mathit{acc}}
\newcommand{\mmkont}{C}
\newcommand{\mstate}{\varsigma}
\newcommand{\mastate}{\hat\varsigma}

\newcommand{\mmarks}{m}
\newcommand{\mmarkset}{\chi}
\newcommand{\mprim}{\mathit{prim}}

\newcommand{\mktab}{\Xi}
\newcommand{\mmktab}{\chi}
\newcommand{\mmemo}{M}
\newcommand{\mpoint}{p}
\newcommand{\mctx}{\mathit{ctx}}
\newcommand{\msctx}{\mathit{vctx}}
\newcommand{\mframe}{\xi}
\newcommand{\mkframe}{\phi}
\newcommand{\mlab}{\ell}
\newcommand{\mtrace}{\pi}

\newcommand{\callcc}{\rackett{call/cc}}
\newcommand{\callcomp}{\rackett{call/comp}}
\newcommand{\abort}{\rackett{abort}}
\newcommand{\dynamicwind}{\rackett{dynamic-wind}}

% Metavariables for Spaces
\newcommand{\Var}{\mathit{Var}}
\newcommand{\Addr}{\mathit{Addr}}
\newcommand{\Expr}{\mathit{Expr}}
\newcommand{\Store}{\mathit{Store}}
\newcommand{\Kont}{\mathit{Kont}}
\newcommand{\SKont}{\widehat{\mathit{Kont}}}
\newcommand{\MKont}{\mathit{MKont}}
\newcommand{\Env}{\mathit{Env}}
\newcommand{\State}{\mathit{State}}
\newcommand{\System}{\mathit{System}}
\newcommand{\Value}{\mathit{Value}}
\newcommand{\Prims}{\mathit{Prims}}

\newcommand{\Context}{\mathit{Context}}
\newcommand{\SContext}{\mathit{VContext}}

\newcommand{\Point}{\mathit{Point}}
\newcommand{\Frame}{\mathit{Frame}}
\newcommand{\Marks}{\mathit{Marks}}
\newcommand{\Markset}{\mathit{Markset}}

\newcommand{\KTab}{\mathit{Stack}}
\newcommand{\MKTab}{\mathit{MStack}}
\newcommand{\Memo}{\mathit{Memo}}

% Other metavariables (metafunction names, etc)
\newcommand{\inject}{\mathit{inject}}
\newcommand{\alloc}{\mathit{alloc}}
\newcommand{\bind}{{\mathcal B}}
\newcommand{\lookup}{{\mathcal L}}
\newcommand{\wn}{\mathit{wn}}
\newcommand{\dom}{\text{dom}}
\newcommand{\onep}{\mathit{One?}}
\newcommand{\reverse}{\mathit{reverse}}
\newcommand{\revapp}{\mathit{rev\text{-}append}}
\newcommand{\finalize}{\mathit{finalize}}
\newcommand{\captureupto}{\mathit{capture\text{-}upto}}
\newcommand{\aborttargets}{\mathit{abort\text{-}targets}}

\newcommand{\startstate}{\mathit{base}}
\newcommand{\tails}[2]{\mathit{tails}_{#1}(#2)}
\newcommand{\reify}{\mathit{reify}}
\newcommand{\reflect}{\mathit{reflect}}
\newcommand{\reachrestrict}{\mathit{unfold}}
\newcommand{\stepextend}{\mathit{unfold}_1}

\newcommand{\domark}{\mathit{mark}}
\newcommand{\domarkset}{\mathit{markset}}
\newcommand{\returns}{{\mathbb R}}
\newcommand{\memoize}{{\mathbb M}}
\newcommand{\approximate}{{\mathbb A}}
\newcommand{\touches}{{\mathcal T}}
\newcommand{\touchesf}{{\mathcal T}_{\mkframe}}
\newcommand{\touchesk}[1]{{\mathcal T}_{#1}}
\newcommand{\reaches}{{\mathcal R}}
\newcommand{\konts}{\mathit{konts}}

\newcommand{\inv}{\mathit{inv}}
\newcommand{\reachable}{\mathit{reach}}
\newcommand{\memo}{\mathit{memo}}
\newcommand{\unroll}[2]{\mathit{unroll}_{#1}(#2)}
\newcommand{\hastail}{\mathit{ht}}
\newcommand{\replacetail}{\mathit{rt}}

% Constructors
\newcommand{\svar}[2][{}]{#2^{#1}}
\newcommand{\sapp}[3][{}]{(#2\ #3)#1}
\newcommand{\slam}[3][{}]{\lambda#1 #2. #3}
\newcommand{\sprim}[1]{#1}
\newcommand{\sreset}[1]{(\texttt{reset}\ #1)}
\newcommand{\sshift}[2]{(\texttt{shift}\ #1. #2)}

\newcommand{\swcm}[3]{(\texttt{wcm}\ #1\ #2\ #3)}
\newcommand{\sccms}{(\texttt{ccm})}
\newcommand{\scmsf}{\texttt{cms-first}}
\newcommand{\scmsl}[2]{(\texttt{cms->list}\ #1\ #2)}

\newcommand{\cons}[2]{(\texttt{cons}\ #1\ #2)}
\newcommand{\smarkset}[1]{(\texttt{mark-set}\ #1)}

\newcommand{\vcont}[1]{\mathbf{cont}(#1)}
\newcommand{\vcomp}[1]{\mathbf{comp}(#1)}
\newcommand{\vclo}[2]{(#1, #2)}
\newcommand{\kar}[2][{}]{\mathbf{ar}#1(#2)}
\newcommand{\kpush}[1]{\mathbf{push}(#1)}
\newcommand{\klet}[1]{\mathbf{lt}(#1)}
\newcommand{\kfn}[2][{}]{\mathbf{fn}#1(#2)}
\newcommand{\kwcm}[1]{\mathbf{km}(#1)}
\newcommand{\kwcmk}[1]{\mathbf{kw1}(#1)}
\newcommand{\kwcmv}[1]{\mathbf{kw2}(#1)}
\newcommand{\kcmsls}[1]{\mathbf{kset}(#1)}
\newcommand{\kcmslk}[1]{\mathbf{kkey}(#1)}

\newcommand{\rt}{\mathbf{rt}}
\newcommand{\krt}[1]{\rt(#1)}
\newcommand{\kmt}{\mathbf{mt}}

\newcommand{\kprompt}[1]{\#(#1)}
\newcommand{\dw}{\mathbf{dw}}
\newcommand{\kdw}[1]{\dw(#1)}
\newcommand{\abrt}{\mathbf{abrt}}
\newcommand{\kabrt}[1]{\abrt(#1)}
\newcommand{\kev}[1]{\mathbf{ev}(#1)}

\newcommand{\mextag}[4]{(\#^{#2}_{#3} #1\ #4)}
\newcommand{\kmatch}[4]{(?^{#2}_{#3} #1\ #4)}
\newcommand{\mkapp}[2]{#1\circ #2}

% Notations
\newcommand{\lfp}{\mathbf{lfp}}
\newcommand{\tpl}[1]{\langle #1 \rangle}
\newcommand{\vect}[1]{\langle #1 \rangle}
\newcommand{\runpost}[1]{\langle #1 \rangle_{\mathit{post}}}
\newcommand{\runhandler}[1]{\langle #1 \rangle_{\mathit{handler}}}
\newcommand{\cstate}[1]{\langle #1 \rangle_{\mathit{call}}}
\newcommand{\set}[1]{\{ #1 \}}
\newcommand{\setbuild}[2]{\{ #1\ :\ #2\}}
\newcommand{\sa}[1]{\widehat{#1}}
\newcommand{\nequiv}{\centernot\equiv}

\newcommand{\many}[1]{\overline{#1}}
\newcommand{\kcons}[2]{#1{\tt :} #2}
\newcommand{\alt}{\mathrel{\mid}}

\newcommand{\snglm}[2]{[#1 \mapsto \set{#2}]}
\newcommand{\extm}[3]{#1[#2 \mapsto #3]}
\newcommand{\joinm}[3]{#1\sqcup[#2 \mapsto #3]}
\newcommand{\joinone}[3]{#1\sqcup[#2 \mapsto \set{#3}]}

\newcommand{\append}[2]{#1 {\tt ++} #2}

%% Relations
\newcommand{\stepto}{\longmapsto}
\newcommand{\pdstepto}{\longmapsto_{\mathit{\Xi{}M}}}
\newcommand{\astepto}{\mathrel{\widehat{\longmapsto}}}
\newcommand{\tred}[1]{\mathrel{\underset{\Pi}{#1}}}
\usepackage[numbers,sectionbib]{natbib}
\usepackage[english]{babel}
% pdfpagescrop for preprint only
%\usepackage[hidelinks,pdfpagescrop={92 62 523 728}]{hyperref}
\usepackage[hidelinks]{hyperref}
\usepackage{cleveref}
\usepackage{placeins,xr}

\externaldocument{paper-jfp}

\newcommand{\SCodePreSkip}{\vskip\abovedisplayskip}
\newcommand{\SCodePostSkip}{\vskip\belowdisplayskip}
\newenvironment{SCodeFlow}{\SCodePreSkip\begin{list}{}{\topsep=0pt\partopsep=0pt%
\listparindent=0pt\itemindent=0pt\labelwidth=0pt\leftmargin=2ex\rightmargin=2ex%
\itemsep=0pt\parsep=0pt}\item}{\end{list}\SCodePostSkip}
\newenvironment{SingleColumn}{\begin{list}{}{\topsep=0pt\partopsep=0pt%
\listparindent=0pt\itemindent=0pt\labelwidth=0pt\leftmargin=0pt\rightmargin=0pt%
\itemsep=0pt\parsep=0pt}\item}{\end{list}}
\newenvironment{RktBlk}{}{}
\definecolor{IdentifierColor}{rgb}{0.15,0.15,0.50}
\definecolor{ParenColor}{rgb}{0.52,0.24,0.14}
\newcommand{\RktSym}[1]{\inColor{IdentifierColor}{#1}}
\newcommand{\RktPn}[1]{\inColor{ParenColor}{#1}}

\newcommand{\Stttextless}{{\fontencoding{T1}\selectfont<}}
\definecolor{ValueColor}{rgb}{0.13,0.55,0.13}
\newcommand{\RktVal}[1]{\inColor{ValueColor}{#1}}

\newcommand{\ifwcm}[1]{}
\begin{document}
\title{The Essence of Summarization\\ in Pushdown Flow Analyses\\
       \textbf{Supplemental Material}}
\author{J. Ian Johnson and David Van Horn}

\section{PDCFA}
\subsection{Metatheory}

Correctness follows directly from the invariants we prove of $\mktab$ and $\mmemo$ in the tabular semantics.
%
Allocation strategies are comparable if they produce equal addresses regardless of the differences in the state representations.
%
More formally, $\alloc$ and $\alloc^*$ are comparable if this implication holds:
\begin{mathpar}
  \inferrule{\mkont \in \unroll{\mktab}{\mkont', \emptyset}}
            {\alloc(\tpl{\mpoint,\mstore,\mkont}) = \alloc^*(\tpl{\mpoint,\mstore,\mkont',\mktab,\mmemo})}
\end{mathpar}

The $\mathit{unroll}$ function interprets what are all the valid continuations that $\mktab$ encodes for a given continuation that contains $\krt{\mctx}$, defined as the greatest fixed point of the following rules:
\begin{mathpar}
  \inferrule{ }{\kmt \in \unroll{\mktab}{\kmt, G}} \quad
  \inferrule{\mkont \in \unroll{\mktab}{\mkont', \emptyset}}{\mkframe:\mkont \in \unroll{\mktab}{\mkframe:\mkont', G}} \\
  \inferrule{\mctx \notin G  \\
             \mkont'' \in \mktab(\mctx) \\
             \mkont \in \unroll{\mktab}{\mkont'', \set{\mctx}\cup G}}
            {\mkont \in \unroll{\mktab}{\krt{\mctx}, G}}  
\end{mathpar}

We add $G$ to protect against unguarded corecursion in order for $\mathit{unroll}$ to be well-defined.
%
Interpreting a function with ill-founded recursion can lead to a table such that $\krt{\mctx} \in \mktab(\mctx)$.
%
The abstractions that we impose on our data might also cause this situation, even though the concrete execution might always terminate.

%%
The different tables we have encode information about execution history that we prove is invariant (\autoref{fig:inv}).
%
The first is that the memo table only contains information about previously seen contexts; we need this to infer that there was at least one call leading to the memoized context such that we can use stack irrelevance to justify skipping to the memoized result.
%
The second, $\phi_{\reachable}$, states that all the calling contexts in the continuation table reach some unrolling of the current state.
%
The final invariant, $\phi_{\memo}$, states that all paths starting at a function call either reach the memoized result, or if the path does not include a return, there is an extension that will.
%
The portion of these paths between call and return hal the calling context's continuation in the tail to justify using stack-irrelevance.

\begin{figure}
  \centering
  \begin{align*}
    \inv(\tpl{\mpoint, \mstore, \mkont, \mktab, \mmemo}) &=
    \dom(\mmemo) \subseteq \dom(\mktab) \\
    &\wedge \forall (\mpoint',\mstore') \in \dom(\mktab), \mkont''
    \in \mktab(\mpoint',\mstore'),
    \mkont' \in \unroll{\mktab}{\mkont'', \emptyset}. \\
    &\qquad \phi_{\reachable} \wedge \phi_{\memo} \\
    \text{where } \phi_{\reachable} &= \exists \mkont'' \in
    \unroll{\mktab}{\mkont,\emptyset}.
    \tpl{\mpoint', \mstore', \mkont'} \stepto^* \tpl{\mpoint, \mstore, \mkont''} \\
    \phi_{\memo} &=
    \forall (\mval, \mstore'') \in \mmemo(\mpoint',\mstore'). \\
    &\qquad\exists \mtrace\equiv\tpl{\mpoint', \mstore', \mkont'} \stepto^* \tpl{\mval,\mstore'',\mkont'}. \hastail(\mtrace,\mkont')
  \end{align*}
  \caption{Table invariants}
\label{fig:inv}
\end{figure}
\begin{lemma}[Table invariants in CESK$\mathit{\Xi{}M}$]\label{lem:tab-inv}
  If $\inv(\mstate)$ and $\mstate \stepto \mstate'$ then $\inv(\mstate')$.
\end{lemma}
\begin{proof}
  By cases on $\stepto$.
  \begin{byCases}
% variable lookup
    \case{\tpl{(\svar\mvar, \menv), \mstore, \mkont, \mktab, \mmemo}
          \stepto
          \tpl{\mval,\mstore,\mkont, \mktab, \mmemo} \text{ if }\mval \in \mstore(\menv(\mvar))}{
     Same continuation and tables, so holds by $\inv(\mstate)$.}
% application
    \case{\tpl{(\sapp{\mexpri0}{\mexpri1}, \menv), \mstore, \mkont, \mktab, \mmemo}
          \stepto
          \tpl{(\mexpri0, \menv), \mstore, \kcons{\kar{\mexpri1,\menv}}{\mkont}, \mktab, \mmemo}}{
    Let $\mctx, \mkont^\circ, \mkont'$ be arbitrary.
    Let $\mkont^*$ be the witness for $\mstate$.
    The next witness is $\kcons{\kar{\mexpri1,\menv}}{\mkont^*}$, by definition of $\mathit{unroll}$.}
% argument evaluation
    \case{\tpl{\mval, \mstore, \kcons{\kar{\mexpr,\menv}}{\mkont}, \mktab, \mmemo}
          \stepto
          \tpl{(\mexpr, \menv), \mstore, \kcons{\kfn{\mval}}{\mkont}, \mktab, \mmemo}}{
    Let $\mctx, \mkont^\circ, \mkont'$ be arbitrary.
    Let $\kcons{\kar{\mexpr,\menv}}{\mkont^*}$ be the witness for $\mstate$.
    The next witness is $\kcons{\kfn{\mval}}{\mkont^*}$, by definition of $\mathit{unroll}$.}
% function call
    \case{\tpl{\mval,\mstore,\kcons{\kfn{\vclo{\slam{\mvar}{\mexpr}}{\menv}}}{\mkont},\mktab,\mmemo}
          \stepto
          \tpl{\mpoint, \mstore', \krt{\mpoint, \mstore'}, \mktab', \mmemo}}{
     where
     $\begin{array}{l}
       \mpoint = (\mexpr, \extm{\menv}{\mvar}{\maddr}) \\
       \mstore' = \joinone{\mstore}{\maddr}{\mval} \\
       \mktab' = \joinone{\mktab}{(\mpoint, \mstore')}{\mkont}
     \end{array}$ \\

    Let $\mctx, \mkont^\circ, \mkont'$ be arbitrary.
    Let $\kcons{\kfn{\mval}}{\mkont^*}$ be the witness for $\mstate$.
    The next witness is $\mkont^*$, by definition of $\mathit{unroll}$.}
% memo-lookup
    \case{\tpl{\mval,\mstore,\kcons{\kfn{\vclo{\slam{\mvar}{\mexpr}}{\menv}}}{\mkont},\mktab,\mmemo}
          \stepto
          \tpl{\mval_\mathit{result}, \mstore'', \mkont, \mktab', \mmemo}
          \text{ if } (\mval_\mathit{result},\mstore'') \in \mmemo(\mpoint,\mstore')}{
        same where clause as previous case.
     
    Let $\mctx, \mkont^\circ, \mkont'$ be arbitrary.
    By definition of $\inv$, and since $\tpl{\mpoint',\mstore',\mkont'} \stepto^* \tpl{\mval_{\mathit{result}}, \mstore'', \mkont'}$.
    For the memo entry to exist, there must have been at least one previous continuation in $\mktab(\mpoint',\mstore')$.
    We use $\inv$ with this continuation, combined with stack irrelevance to produce the trace involving $\mkont'$.
    }
% return
    \case{\tpl{\mval, \mstore, \krt{\mpoint,\mstore'}, \mktab, \mmemo}
          \stepto
          \tpl{\mval, \mstore, \mkont, \mktab, \joinone{\mmemo}{(\mpoint, \mstore')}{(\mval,\mstore)}}
          \text{ if } \mkont \in \mktab(\mpoint, \mstore')}{
    Let $\mctx, \mkont^\circ, \mkont'$ be arbitrary.
    $\mkont'$ is an acceptable witness for the first property.
    The path we construct to $\mstate'$ is sufficient for the second property.}
  \end{byCases}
\end{proof}

In order to prove this, we need a lemma that allows us to plug in any continuation for memoized results.
%
When traces (sequences of states such that adjacent states are related by $\stepto$) have a common tail, we can replace it with anything.

\begin{lemma}[Stack irrelevance in CESK]\label{lem:irrelevance}
  $\hastail(\mtrace,\mkont)$ implies $\replacetail(\mtrace,\mkont,\mkont')$ is a valid trace.
\end{lemma}
\begin{proof}
  By induction on $\mtrace$. Base cases trivially hold, and induction step follows from cases on $\mstate \stepto \mstate'$ and definition of $\replacetail$.
\end{proof}

\begin{mathpar}
  \inferrule{ }{\hastail(\epsilon,\mkont)} \quad
  \inferrule{ }{\hastail(\tpl{\mpoint,\mstore,\append{\mkont'}{\mkont}},\mkont)} \quad
  \inferrule{\hastail(\mtrace\mstate,\mkont) \quad
             \mstate \stepto \mstate' \quad
             \hastail(\mstate',\mkont)}
            {\hastail(\mtrace\mstate\mstate',\mkont)}
\end{mathpar}

\begin{align*}
  \replacetail(\tpl{\mpoint,\mstore,\append{\mkont'}{\mkont}},\mkont,\mkont'') &= \tpl{\mpoint,\mstore,\append{\mkont'}{\mkont''}} \\
  \replacetail(\epsilon,\mkont,\mkont'') &= \epsilon \\
  \replacetail(\mtrace\mstate,\mkont,\mkont') &= \replacetail(\mtrace,\mkont,\mkont')\replacetail(\mstate,\mkont,\mkont') \\
\end{align*}

\begin{theorem}[Correctness of Concrete Summarization]\label{thm:concrete-tabular}
  The standard semantics has a skipping bisimulation with the tabular semantics given comparable fresh allocation strategies.
\end{theorem}
\begin{proof}[Sketch]
  The tabular semantics skipping simulates that standard semantics by appealing to the table invariants to show the long ``recomputation'' path.
  Crucially, since the memo table is only set once before being used to bypass to the result, there must be only one possible result in the standard semantics. Otherwise, the tabular semantics would miss out on the other results for subsequent calls with the same context.
  The fresh allocation strategy ensures that relation is deterministic and thus function calls only reduce to one result.

  The standard semantics skipping simulates the tabular semantics because ``recomputing'' paths always have the calling stack as its tail, so we can step forward on the invariant's inductively built path until the result.
\end{proof}

Finally, we can show that the reductions we find between states paired with the shared tables define a sound and complete abstraction of the CESK machine with a comparable allocation strategy.


\begin{theorem}\label{thm:concrete-sr}
  Tabular semantics simulates the standard semantics given fresh allocation.
\end{theorem}
\begin{proof}[Sketch]
  We have the invariant that the continuation and meta-continuation have unique unrollings in addition to the table invariants from paper's section 3,
  since the distinct $\mstore$ at push and capture time makes entries in $\mktab$ unique, regardless of $\mmktab$.
\end{proof}

\noindent{}Proof of paper's theorem ``Correctness of PDCFA''
\begin{proof}[Sketch]
  The invariants on the tables still hold. 
  The widening to share $\mmemo$ and $\mktab$ amongst all explored states reifies the search state into a $\sa{System}$. We can add a set of edges simply.
  If we add $\mmemo$ and $\mktab$ to all states in these edges, we get a reduction relation comparable to the unwidened tabular semantics.
  We prove that the infinite unrolling of the standard reduction relation stuttering bisimulates the relation reified from an infinite stepping of the widened tabular semantics.
  This only holds if the state selection from the frontier is fair, since unfair choices can leave some function calls unresolved.
  Formally,
  \begin{equation*}
    \setbuild{\mtrace}{\mtrace : i \le \omega, \mstate_0 \longmapsto^i \mstate} \simeq {\mathcal F}(\mstate_0)^\omega
  \end{equation*}
  where $\simeq$ is TODO
by coinduction and appeals to the table invariants.
  If a new function result is found after a call already used the memo table, the context of the memo user is used after memoizing, so it looks like that result was part of the original memo table lookup.
  The fair selection guarantees that all function results will be eventually discovered, so the memo table appeals eventually catch up to the found results.
\end{proof}

\begin{theorem}[No dangling pointers]\label{thm:gc-pointers}
  All addresses in $\Gamma(\mstate)$ are in the domain of $\Gamma(\mstate).\mstore$.
\end{theorem}

\begin{theorem}[Garbage is irrelevant]\label{thm:gc-concrete}
  If $\mstate \stepto^n \mstate'$ and $\mstate \stepto^n \mstate''$ then $\Gamma(\mstate') = \Gamma(\mstate'')$.
\end{theorem}

\begin{theorem}[Abstract GC is sound for garbage-free states]\label{thm:gc-sound}
  If $\mstate \stepto \mstate'$ and $\alpha(\Gamma(\mstate)) \sqsubseteq \mastate$ then there is a $\mastate'$ such that $\mastate \astepto \mastate'$ and $\alpha(\Gamma(\mstate')) \sqsubseteq \mastate'$.
\end{theorem}

This final proposition is not in the original work on abstract GC; they prove soundness with respect to a concrete semantics that performs GC after every step.
%
The moral correctness of the abstraction is that it never under-approximates \emph{relevant} portions of the state space --- in this case, the reachable subset of the store.
%
The definitions of $\alpha$, $\astepto$ and $\sqsubseteq$ are the simple pointwise abstractions of addresses through states and the reduction relation, and finally a lifted subset relation between states and their components, respectively.
%
Both $\alpha$ and $\sqsubseteq$ are parameterized on the choice of $\Addr$, for which the following soundness criterion should hold:
\begin{mathpar}
  \inferrule*[right={[Sound allocation]}]{\alpha(\mstate) \sqsubseteq \mastate}{\alpha(\alloc(\mstate)) \sqsubseteq \widehat{\alloc}(\mastate)}
\end{mathpar}

\section{Shift/reset}
\noindent{}Proof of paper's theorem 4 follows the same line of reasoning as the correctness of PDCFA.


% \noindent{}Proof of \autoref{thm:pdcfa-tabular}, which is more formally stated as (tabular $\pdstepto$, CESK $\stepto$): There is a well-founded equivalence bisimulation between $\pdstepto$ (quotiented to states satisfying $\inv$) and $\tred{\stepto}$.
% We define equivalence ($\simeq$) as
% \begin{mathpar}
%   \inferrule{\mkont \in \unroll{\mktab}{\mkont'} \\ \mkont \nequiv \kcons{\kfn{\mval}}{\mkont''}}
%             {\mtrace\tpl{\mpoint,\mstore,\mkont} \simeq \tpl{\mpoint,\mstore,\mkont,\mktab,\mmemo}} \\
% %
%   \inferrule{\mkont \in \unroll{\mktab}{\mkont'} \\
%              ((\mexpr,\menv[\mvar \mapsto \maddr]), \mstore\sqcup[\maddr \mapsto \set{\mval}]) \notin \dom(\mmemo)}
%             {\mtrace\tpl{\mval,\mstore,\kcons{\kfn{\slam{\mvar}{\mexpr}, \menv}{\mkont}}} \simeq \tpl{\mval,\mstore,\kcons{\kfn{\slam{\mvar}{\mexpr},\menv}}{\mkont'},\mktab,\mmemo}} \\
% %
%   \inferrule{\tpl{\mval', \mstore', \mkont} \notin \mtrace'' \\
%              \mkont \in \unroll{\mktab}{\mkont''} \\
%              ((\mexpr,\menv[\mvar \mapsto \maddr]), \mstore\sqcup[\maddr \mapsto \set{\mval}]) \in \dom(\mmemo)}
%             {\mtrace'\tpl{\mval, \mstore, \kcons{\kfn{\slam{\mvar}{\mexpr}, \menv}}{\mkont}}\mtrace''
%               \simeq \tpl{\mval, \mstore, \kcons{\kfn{\slam{\mvar}{\mexpr}, \menv}}{\mkont''}, \mktab, \mmemo}}
% \end{mathpar}
% This means that states which ``line up'' regardless of history are related, and that states that follow a function call that has not yet completed are related to the function call state where we already have the memoized results.
% %
% Each step in this case will get ``closer to'' the result to memoize, as given by the length of the path we get from the invariant on the memo table.
% \begin{proof}
%   Let $\mtrace,\mtrace',\mstate_\Xi$ be arbitrary such that $\mtrace \simeq \mstate_\Xi$ and $\mtrace \tred{\stepto} \mtrace'$.
%   By cases on $\mtrace \simeq \mstate_\Xi$:
%   \begin{byCases}
%     \case{\text{First rule}}{One step related by simple case analysis.}
%     \case{\text{Second rule}}{One step related by definitions of $\mathit{unroll}$, $\stepto$ and $\pdstepto$.}
%     \case{\text{Third rule}}{
%     By cases on $\mtrace\mstate \tred{\stepto} \mtrace\mstate\mstate'$:
%     \begin{byCases}
%       \case{\mtrace\tpl{(\svar\mvar, \menv), \mstore, \mkont} \stepto
%         \mtrace\mstate\tpl{\mval,\mstore,\mkont}}{...}
%       % application
%       \case{\mtrace\tpl{(\sapp{\mexpri0}{\mexpri1}, \menv), \mstore,
%           \mkont} \stepto \mtrace\mstate\tpl{(\mexpri0, \menv),
%           \mstore, \kcons{\kar{\mexpri1,\menv}}{\mkont}}}{...}
%       % argument evaluation
%       \case{\mtrace\tpl{\mval, \mstore,
%           \kcons{\kar{\mexpr,\menv}}{\mkont}} \stepto
%         \mtrace\mstate\tpl{(\mexpr, \menv), \mstore,
%           \kcons{\kfn{\mval}}{\mkont}}}{...}
%       % function call
%       \case{\mtrace\tpl{\mval,\mstore,\kcons{\kfn{\vclo{\slam{\mvar}{\mexpr}}{\menv}}}{\mkont}}
%         \stepto \mtrace\mstate\tpl{(\mexpr,
%           \extm{\menv}{\mvar}{\maddr}),
%           \joinone{\mstore}{\maddr}{\mval}, \mkont}}{...}
%     \end{byCases}}
%   \end{byCases}
%   By \autoref{lem:tab-inv}
  
%   Soundness is a corollary of \autoref{lem:tab-inv}.
% \end{proof}

% \noindent{}Proof of \autoref{thm:global}.
% \begin{proof}
% %% TODO
%   The invariants of the memo and k-tables are independent of the states that carry them, so lifting them to a global position does not affect their information.
% \end{proof}

% \noindent{}Proof of \autoref{thm:refinement}.
% \begin{proof}
%   todo
% \end{proof}

% \noindent{}Proof of \autoref{thm:cfa2}.
% \begin{proof}
%   todo
% \end{proof}

% \noindent{}Proof of \autoref{thm:concrete-sr}.
% \begin{proof}
%   todo
% \end{proof}

% \noindent{}Proof of \autoref{thm:sound-sr}
% \begin{proof}
%   todo
% \end{proof}

\end{document}