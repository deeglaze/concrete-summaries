\subsection{Correctness}

Our correctness claim is that a PDCFA system reflects the subset of the CESK reduction semantics that is reachable from the initial state.
%
We reify the reduction relation from a system in the following way:

\begin{align*}
  \reify &: \mathit{System} \to \wp(\mathit{State}_\mathit{CESK} \times \mathit{State}_\mathit{CESK}) \\
  \reify(F, R, \mktab, \mmemo) &=
  \setbuild{(\tpl{\mpoint,\mstore,\append{\mkont}{\mkont''}},
             \tpl{\mpoint',\mstore',\append{\mkont'}{\mkont''}})}
           {\\ &\phantom{= \{}(\tpl{\mpoint^*,\mstore^*,\mkont^*}, \tpl{\mpoint',\mstore',\mkont'}) \in R,
            \\ &\phantom{= \{}\text{if } \mpoint^*\equiv\mval \text{ and } \mkont^* \equiv \krt{\mctx} \text{ then }
            \\ &\phantom{= \{else }\text{let } (\tpl{\mpoint,\mstore,\mkont},\tpl{\mpoint^*,\mstore^*,\mkont^*}) \in R,
            \\ &\phantom{= \{}\text{else let } \tpl{\mpoint,\mstore,\mkont} = \tpl{\mpoint^*,\mstore^*,\mkont^*}
            \\ &\phantom{= \{}\mkont'' \in \tails{\mktab}{\mkont}} \\
\text{where } \tails{\mktab}{\kmt} &= \set{\kmt} \\
              \tails{\mktab}{\krt{\mctx}} &= \setbuild{\mkont' \in \unroll{\mktab}{\mkont}}{\mkont \in \mktab(\mctx)} \\
              \tails{\mktab}{\kcons{\mkframe}{\mkont}} &= \tails{\mktab}{\mkont}
\\[2pt]
              \unroll{\mktab}{\mkont} &= \setbuild{\append{\mkont}{\mkont'}}{\mkont' \in \tails{\mktab}{\mkont}}
\\[2pt]
              \append{\kmt}{\mkont} &= \append{\krt{\mctx}}{\mkont} = \mkont \\
              \append{\kcons{\mkframe}{\mkont}}{\mkont'} &= \kcons{\mkframe}{(\append{\mkont}{\mkont'})}
\end{align*}

Any return edge (\ie, the left state has a value and a $\krt{\mctx}$ continuation) is administrative, so the reification stutters those out by linking the predecessors of return states with their successors.
%
The $\mathit{unroll}$ set interprets what are all the valid continuations that $\mktab$ encodes for a given continuation that contains $\krt{\mctx}$, and $\mathit{tails}$ is defined mutually to produce all continuations that could be unrolled from the tail of a truncated continuation.

For a pushdown analysis, we can simply take the least fixed-point of ${\mathcal F}(\mastate_0)$, but in general we can only say a witnessed reduction after $n$ steps, in one case, will in the other case witness the same reduction in some $m$ steps.
%
In the limit, the CESK reduction relation unfolds to the reification of the PDCFA system, and vice versa.
%
One small caveat is that state selection from $F$ must be fair: for all states $\mastate \in F$ there is some $m \in \nat$ such that $\mastate$ not in $F$ component of ${\mathcal F}(\mastate_0)^m(F,R,\mktab,\mmemo)$, which means $\mastate$ was selected and processed in one of those $m$ steps.
%
A queue representation of $F$ suffices to guarantee fairness.

\begin{theorem}[Correctness of Summarization]\label{thm:concrete-tabular}
  $\forall \mexpr$ closed, let $\mstate_0 = \tpl{\mexpr,\bot\bot,\kmt}$ in
  $\forall n \in \nat,\mstate,\mstate' \in \State_\mathit{CESK}$:
  \begin{itemize}
  \item{if $(\mstate,\mstate') \in \reachrestrict(\mstate_0,\stepto_\mathit{CESK},n)$ then
      $\exists m \in \nat. (\mstate,\mstate') \in \reify({\mathcal F}(\mstate_0)^m(\mathit{sys}_0))$}
  \item{
      if $(\mstate,\mstate') \in \reify({\mathcal F}(\mstate_0)^n(\mathit{sys}_0))$ then
      $\exists m \in \nat. (\mstate,\mstate') \in \reachrestrict(\mstate_0,\stepto_\mathit{CESK},m)$}
  \end{itemize}
  given comparable allocation strategies and fair state selection.
\end{theorem}

We unfold a reduction relation from a starting state in the following way:
\begin{align*}
  \reachrestrict(\mstate_0, \stepto, 0) &= \setbuild{(\mstate_0,\mstate)}{\mstate_0 \stepto \mstate} \\
  \reachrestrict(\mstate_0, \stepto, n+1) &= \stepextend(\reachrestrict(\mstate_0,\stepto,n)) \\
  \textit{where } \stepextend(R) &= R \cup \setbuild{(\mstate,\mstate')}{(\_,\mstate) \in R, \mstate \stepto \mstate'}
\end{align*}
For the rest of this section, we will leave the $\mstate_0$ and $\stepto_\mathit{CESK}$ arguments to $\mathit{unfold}$ implicit.

Allocation strategies $\alloc$ and $\alloc^*$ are comparable if they produce equal addresses regardless of the differences in the state representations:
\begin{mathpar}
  \inferrule{\mkont \in \unroll{\mktab}{\mkont'}}
            {\alloc(\tpl{\mpoint,\mstore,\mkont}) = \alloc^*(\tpl{\mpoint,\mstore,\mkont',\mktab,\mmemo})}
\end{mathpar}

We build toward a proof of correctness by giving invariant characterizations of the $\mktab$ and $\mmemo$ tables (\autoref{fig:inv}), which further depend on a ``context irrelevance'' lemma about the CESK machine.
%
Our development is partially verified in Coq in \texttt{model.v}, available with the Redex models; propositions marked with ${}^*$ are mechanically checked.

\begin{lemma}[Context irrelevance${}^*$]\label{lem:stack-irrelevance}
  For all CESK traces $\mtrace$ and continuations $\mkont$ such that $\hastail(\mtrace,\mkont)$,
  for any $\mkont'$, $\replacetail(\mtrace,\mkont,\mkont')$ is also a valid trace.
\end{lemma}
Where $\hastail$ (``has tail'') is
\begin{mathpar}
  \inferrule{ }{\hastail(\epsilon,\mkont)} \quad
  \inferrule{ }{\hastail(\tpl{\mpoint,\mstore,\append{\mkont'}{\mkont}},\mkont)} \\
  \inferrule{\hastail(\mtrace\mstate,\mkont) \quad
             \mstate \stepto \mstate' \quad
             \hastail(\mstate',\mkont)}
            {\hastail(\mtrace\mstate\mstate',\mkont)}
\end{mathpar}
and $\replacetail$ (``replace tail'') is
\begin{align*}
  \replacetail(\tpl{\mpoint,\mstore,\append{\mkont'}{\mkont}},\mkont,\mkont'') &= \tpl{\mpoint,\mstore,\append{\mkont'}{\mkont''}} \\
  \replacetail(\epsilon,\mkont,\mkont'') &= \epsilon \\
  \replacetail(\mtrace\mstate,\mkont,\mkont') &= \replacetail(\mtrace,\mkont,\mkont')\replacetail(\mstate,\mkont,\mkont') \\
\end{align*}
\begin{proof}
  By induction on $\mtrace$ and cases on $\stepto$.
\end{proof}
The first invariant is that the memo table only contains information about previously seen contexts; we need this to infer that there was at least one call leading to the memoized context such that we can use stack irrelevance to justify skipping to the memoized result.
%
Second, the continuations that weave through $\mktab$ have successively larger contexts, and of course the contexts in continuations are also mapped in $\mktab$.
%
Additionally, the current continuation will be able to satisfy the order property if it is stored in $\mktab$.
%
Third, there is a path from the starting state of the current continuation to the current state.
%
Fourth, similar paths exist for all continuations stored in $\mktab$.
%
Finally, there is some continuation that stays in the tail of the trace of a memoization entry from starting context to final results (so we can interchange continuations with context irrelevance to justify memo table uses).
%
The properties that append $\kmt$ ensure that the continuations are proper CESK continuations. We could quantify over all continuations instead of using $\kmt$, but the context irrelevance lemma makes that unnecessary.
The third and fourth properties allow us to build up paths piecemeal as we need them.
%
%The second, $\phi_{\reachable}$, states that all the calling contexts in the continuation table reach some unrolling of the current state.
%
%The final invariant, $\phi_{\memo}$, states that all paths starting at a function call either reach the memoized result, or if the path does not include a return, there is an extension that will.
%
%The portion of these paths between call and return hal the calling context's continuation in the tail to justify using stack-irrelevance.

\begin{figure}
  \centering
  \begin{align*}
    \startstate(\kmt) &= \tpl{\mexpr_\mathit{initial}, \bot, \bot, \kmt} \\
    \startstate(\krt{\mctx}) &= \tpl{\mctx, \kmt} \\    
    \startstate(\kcons{\mkframe}{\mkont}) &= \startstate(\mkont)
  \end{align*}
  \begin{mathpar}
    \inferrule{ }{\klectx{\mktab}{\kmt}{\mctx}} \quad
    \inferrule{\mstore \sqsubseteq \mstore' \quad (\mpoint,\mstore) \in \dom(\mktab)}{\klectx{\mktab}{\krt{\mpoint,\mstore}}{(\mpoint,\mstore')}} \quad
    \inferrule{\klectx{\mktab}{\mkont}{\mctx}}{\klectx{\mktab}{\kcons{\mkframe}{\mkont}}{\mctx}}
\\
    \inferrule{
        \dom(\mmemo) \subseteq \dom(\mktab) \\ % domain containment
    % order in \Xi.
        \forall \mctx \in \dom(\mktab), \mkont' \in \mktab(\mctx). \klectx{\mktab}{\mkont'}{\mctx} \quad
    % order on current cont
        \klectx{\mktab}{\mkont}{(\mpoint,\mstore)} \\
    % path to current kont
        \startstate(\mkont) \stepto^* \tpl{\mpoint, \mstore,\append{\mkont}{\kmt}} \\
    % ktable meaning
        \forall (\mpoint',\mstore') \in \dom(\mktab), \mkont'
        \in \mktab(\mpoint',\mstore'). \\
        \quad\startstate(\mkont') \stepto^* \tpl{\mpoint',\mstore',\append{\mkont'}{\kmt}} \\
    % memo table meaning
      \forall(\mpoint',\mstore') \in \dom(\mmemo), (\mval,\mstore'') \in \mmemo(\mpoint',\mstore').
      \\ \quad
       \exists \mkont, \mtrace\equiv\tpl{\mpoint',\mstore',\mkont}\stepto^* \tpl{\mval,\mstore'',\mkont}.
         \hastail(\mtrace,\mkont)
      }
    {\inv(\tpl{\mpoint, \mstore, \mkont, \mktab, \mmemo})}
  \end{mathpar}
  \caption{Table invariants}
\label{fig:inv}
\end{figure}
\begin{lemma}[Table invariants in CESK$\mathit{\Xi{}M}^*$]\label{lem:tab-inv}
  If $\inv(\mstate)$ and $\mstate \stepto \mstate'$ then $\inv(\mstate')$.
\end{lemma}
\begin{proof}
  By cases on $\stepto$.
\end{proof}
As the later developments in this paper are of a similar character, we will describe correctness claims as a system reflecting a relation that has a bisimulation with the original semantics.
%
We argue the correctness theorem by induction on $n$:
\begin{proof}
  Base case, first bullet: $m = 2$, since $m = 1$ adds the initial state to the frontier, and in $m = 2$ it is the only choice to process. By cases on $\stepto$ the reductions are in the reification. \\
  Base case, second bullet: vacuously true, since the relation is empty. \\
  Induction step, first bullet: 
%IH: $\forall (\mstate,\mstate') \in \mathit{unfold}(\mstate_0,\stepto,n). \exists m. (\mstate,\mstate') \in \reify({\mathcal F}(\mstate_0)^m(\mathit{sys}_0))$
given $(\mstate,\mstate') \in \mathit{unfold}(n+1) = \mathit{unfold}_1(\mathit{unfold}(n))$,
by cases on $(\mstate,\mstate') \in \mathit{unfold}(n)$ or $(\mstate,\mstate') \in \setbuild{(\mstate,\mstate')}{(\_,\mstate) \in \mathit{unfold}(n), \mstate \stepto \mstate'}$:
\\
Case 1: Holds by IH.
\\
Case 2: there is some $\mstate''$ such that $(\mstate'',\mstate) \in \mathit{unfold}(n)$.
        By IH, there is an $m$ such that (let $\mathit{sys} = {\mathcal F}(\mstate_0)^m(\mathit{sys}_0)$) $(\mstate'',\mstate) \in \mathit{reify}(\mathit{sys})$.
\\
        So $(\mstate'',\mstate) \equiv (\tpl{\mpoint, \mstore, \append{\mkont}{\mkont''}},\tpl{\mpoint',\mstore',\append{\mkont'}{\mkont''}})$ where $\mkont'' \in \tails{\mktab}{\mkont}$.

   \noindent By cases on $\tpl{\mpoint',\mstore',\mkont'} \in \mathit{sys}.F$:
\\
    Case in: let $\ell$ be the fairness witness. We follow the same argument for $\mathit{sys}={\mathcal F}(\mstate_0)^{m+\ell}(\mathit{sys}_0)$ as
              the following case.
\\
    Case not in: it must be that $\tpl{\mpoint',\mstore',\mkont'}$ was selected and ``processed.''
                 By cases on $\mstate \stepto \mstate'$:
\\
        Case $\tpl{(\mvar,\menv),\mstore',\append{\mkont'}{\mkont''}} \stepto \tpl{\mval,\mstore',\append{\mkont'}{\mkont''}}$ where $\mval \in \mstore(\menv(\mvar))$:
        $\mkont''$ is irrelevant, thus $(\mstate,\mstate') \in \mathit{reify}(\mathit{sys})$
\\
        Case $\tpl{((\mexpri0\ \mexpri1),\menv),\mstore',\append{\mkont'}{\mkont''}} \stepto \tpl{(\mexpri0, \menv),\mstore',\append{\mkont'}{\mkont''}}$,
        thus $(\mstate,\mstate') \in \mathit{reify}(\mathit{sys})$
\\
        Case $p'\equiv\mval, \append{\mkont'}{\mkont''}\equiv\kcons{\mkframe}{\mkont^*}$ (that is, $\mkont' \equiv \kcons{\mkframe}{\mkont^{**}}$ and $\append{\mkont^{**}}{\mkont''}=\mkont^*$ or $\mkont'\equiv\krt{\mctx}$ and $\mkont''=\kcons{\mkframe}{\mkont^*}$)
         For $\mkont'$ case 2: $\mkont''$ must be unrolled from $\krt{\mctx} \in \mathit{sys}.\mktab$.
                         Thus there is a $\kcons{\mkframe}{\mkont^{**}} \in \mathit{sys}.\mktab(\mctx)$ such that $\mkont^* \in \unroll{\mathit{sys}.\mktab}{\mkont^{**}}$
                         \\
         Case $\mkframe\equiv\kar{\mexpr,\menv}$, $\mstate' \equiv \tpl{(\mexpr,\menv),\mstore',\mkont^*}$:
          By definition of ${\mathcal F}$, $(\tpl{\mpoint',\mstore',\mkont'}, \tpl{(\mexpr,\menv),\mstore',\kcons{\kfn{\mval}}{\mkont^{**}}}) \in \mathit{sys}.R$, so $(\mstate,\mstate') \in \mathit{reify}(\mathit{sys})$.
\\
         Case $\mkframe\equiv\kfn{\vclo{\slam{x}{\mexpr}}{\menv}}$: Similar.
\\
%%

%%
\noindent  Induction step, second bullet:
Let $\mathit{sys} = {\mathcal F}(\mstate_0)^n(\mathit{sys}_0)$.
Let $\mathit{sys}' = {\mathcal F}(\mathit{sys})$
%IH: $\forall (\mstate,\mstate') \in \mathit{reify}(\mathit{sys}). \exists m. (\mstate,\mstate') \in \mathit{unfold}(\mstate_0,\stepto,m)$
Given $(\mstate,\mstate') \in \mathit{reify}(\mathit{sys}')$,
%WTS $\exists m. (\mstate,\mstate') \in \mathit{unfold}(\mstate_0,\stepto,m)$.
%
we know $(\mstate,\mstate') \equiv (\tpl{\mpoint,\mstore,\append{\mkont}{\mkont''}},\tpl{\mpoint',\mstore',\append{\mkont'}{\mkont''}})$ where
\\$(\tpl{\mpoint,\mstore,\mkont},\tpl{\mpoint',\mstore',\mkont'}) \in \mathit{sys}'.R$, if $p \equiv \mval$ then $\mkont \nequiv \krt{\mctx}$ and $\mkont'' \in \tails{\mathit{sys}'.\mktab}{\mkont}$.

\noindent By cases on $\mstate$ initial:
Case $\mstate = \mstate_0$: a sufficient $m$ is 0.
\\
Case else: it must be that there is a state that steps to $\tpl{\mpoint,\mstore,\mkont}$, with the edge represented in $\mathit{sys}.R$.
%
 $(\tpl{\mpoint^*,\mstore^*,\mkont^*},\tpl{\mpoint,\mstore,\mkont}) \in \mathit{sys}.R$.
%
 By IH, for fixed arbitrary $\mkont_o \in \tails{\mathit{sys}.\mktab}{\mkont^*}$, there is an $m$ such that
 $(\tpl{\mpoint^*,\mstore^*,\append{\mkont^*}{\mkont_o}},\tpl{\mpoint,\mstore,\append{\mkont}{\mkont_o}}) \in \mathit{unfold}(m)$.
 We know that the context of $\mkont$ must be in the domain of $\mathit{sys}.\mktab$ due to the invariant.
\\
\newcommand{\ctxof}{\mathit{ctx\text{-}of}}
 \textbf{The intricate part of the proof:}
 If $\mkont''$ uses any truncated continuations from $\mathit{sys}'.\mktab$ that are not in $\mathit{sys}.\mktab$, they must all be in the entry for the context in $\mkont$, and they must all be at the top of $\mkont''$ with no other truncated continuations from other contexts' entries (due to the order invariant).
 In other words, $\mkont''= \append{\append{\mkont_0}{\append{\ldots}{\mkont_n}}}{\mkont_o}$ for some $\vec{\mkont} \in \mathit{sys}'.\mktab(\ctxof(\mkont))^*$ and $\mkont_o \in \tails{\mathit{sys}.\mktab}{\mkont}$.
We mean $\ctxof$ to extract the context stored in a continuation's $\mathbf{rt}$.
%
If the continuation does not have an $\mathbf{rt}$, then $\mkont''$ must be $\kmt$, and the argument becomes trivial.
 By induction on this list of $\vec{\mkont}$, we can use the invariants to build a path piecemeal.
\\
 Specifically we argue \emph{a fortiori} the following:
\\
 For all $\vec{\mkont} \in \mathit{sys}'.\mktab(\ctxof(\mkont))^*$,
 all $(\mstate,\mstate') \in \reify(\mathit{sys}')$,
 all $\mkont_o \in \tails{\mathit{sys}.\mktab}{\mkont}$,
 and all $\mpoint^*,\mstore^*,\mkont^*,\mpoint',\mstore',\mkont',$ if $(\mstate,\mstate')$ equals
\\
$(\tpl{\mpoint^*,\mstore^*,\append{\mkont^*}{\append{\vec{\mkont}}{\mkont_o}}},\tpl{\mpoint',\mstore',\append{\mkont'}{\append{\vec{\mkont}}{\mkont_o}}})$ and
\\
 $\tails{\mathit{sys}'.\mktab}{\mkont^*} = \tails{\mathit{sys}'.\mktab}{\mkont}$, then there is some natural $m$, such that $(\mstate,\mstate') \in \mathit{unfold}(m)$.
\\
\textit{Proof.} The argument follows by induction on $\vec{\mkont}$:
\\
Base case ($\vec{\mkont}$ empty): We use the (outer) induction hypothesis with $\mkont_o$ to get an $m$. Then since $\mstate \stepto \mstate'$, we conclude $(\mstate,\mstate') \in \mathit{unfold}(m+1)$.
\\
Induction step ($\vec{\mkont} \equiv \mkont_0 : \vec{\mkont}'$):
By the invariant and context irrelevance, there is a path $\mtrace\equiv\tpl{\ctxof(\mkont_0), \append{\vec{\mkont}}{\mkont_o}} \stepto^* \mstate_\mathit{base}$ where $\mstate_\mathit{base} = \tpl{\ctxof(\mkont),\append{\mkont_0}{\append{\vec{\mkont}}{\mkont_o}}}$.
Also by the invariant, there is a path $\mtrace' \equiv \mstate_\mathit{base} \stepto^* \mstate$.
We use the last edge of $\mtrace$ and the induction hypothesis to get an $m$ to unfold through the second path to show $(\mstate,\mstate') \in \mathit{unfold}(m + |\mtrace'| + 1)$.
%We case split on whether $\mkont_0$ has a context equal to the context of $\mkont$.

% By cases on edge in $\mathit{sys}.R$ or in $\mathit{sys}'.R$ (and not in $\mathit{sys}.R$).
% \\
% Case in: $\mkont''$ is irrelevant because it's in the tail, and there must be some continuation in $\tails{\mathit{sys}.\mktab}{\mkont}$ by $\inv$.
%          Thus holds by IH.
% \\
% Case not in:
%   It must be that $\tpl{\mpoint,\mstore,\mkont}$ was the selected state, and that
%   $\tpl{\mpoint,\mstore,\mkont,\mathit{sys}.\mktab,\mathit{sys}.\mmemo} \stepto \tpl{\mpoint',\mstore',\mkont',\mktab',\mmemo'}$ where $\mktab' \sqsubseteq \mathit{sys}'.\mktab$ and $\mmemo' \sqsubseteq \mathit{sys}'.\mmemo$.

%   By cases on this reduction:
% \\
%   Case $\tpl{(\mvar,\menv),\mstore,\mkont,\_,\_} \stepto \tpl{\mval,\mstore,\mkont,\_,\_}$ where $\mval \in \mstore(\menv(\mvar))$:
% \\
%   Case $\tpl{((\mexpri0\ \mexpri1),\menv),\mstore,\mkont,\_,\_} \stepto \tpl{(\mexpri0,\menv),\mstore,\mkont,\_,\_}$:
% \\
%   Case $\tpl{\mval,\mstore,\kcons{\kar{\mexpr,\menv}}{\mkont},\_,\_} \stepto \tpl{(\mexpr,\menv),\mstore,\kcons{\kfn{\mval}}{\mkont},\_,\_}$:
% \\
%   Case $\tpl{\mval,\mstore,\kcons{\kfn{\vclo{\slam{\mvar}{\mexpr}}{\menv}}}{\mkont},\mktab,\mmemo} \stepto \tpl{\mctx,\krt{\mctx},\mktab',\mmemo}$
%   where $\maddr = \alloc^*(\mstate)$, $\menv' = \extm{\menv}{\mvar}{\maddr}$, $\mstore' = \joinone{\mstore}{\maddr}{\set{\mval}}$, $\mctx = ((\mexpr,\menv'),\mstore')$, and $\mctx \notin \dom(\mmemo)$:
% \\
%   Split on $\tpl{\mpoint,\mstore,\mkont}$ the initial state or
%            there is a $\tpl{\mpoint^*,\mstore^*,\mkont^*}$ such that $(\tpl{\mpoint^*,\mstore^*,\mkont^*}, \tpl{\mpoint,\mstore,\mkont}) \in \mathit{sys}.R$.
% \\
%   Case initial: $m$ is 0.
% \\
%   Case else:
%   \textbf{TODO:}
%    Need to show there is a path using $\mkont''$ whereever $\tpl{\mpoint,\mstore,\mkont}$ is, and the length should be an upper bound for $m$.
%    This brings in the table invariants and the order invariant which allows us to get to the starting point for 
%    $\tpl{\mpoint^*,\mstore^*,\mkont^*}$ and replace with the latest frame to get to where we are.
\end{proof}

To reiterate, this correctness proof is strong enough for any allocation strategy, including fresh (for concrete execution) and finite (for ``pushdown analysis'').
%
The finite case is special since the $\System$ space becomes finite, which means there is some $m$ such that ${\mathcal F}^m(\mastate_0)$ is a fixed point, by Kleene's fixed point theorem.
%
In other words, the fixed point soundly and completely represents all paths reachable from the initial state in the CESK machine with an unbounded stack but finite address space.
%
This form of the CESK machine has a shallow translation into a pushdown system, but we needn't go there.