\subsection{Metatheory}

Our correctness claim is that a PDCFA system reflects the subset of the CESK reduction semantics that is reachable from the initial state.
%
We reify the reduction relation from a system in the following way:

\begin{align*}
  \reify &: \sa{System} \to \wp(\sa{State}_\mathit{CESK} \times \sa{State}_\mathit{CESK}) \\
  \reify(F, R, \mktab, \mmemo) &=
  \setbuild{(\tpl{\mpoint,\mstore,\append{\mkont}{\mkont''}},
             \tpl{\mpoint',\mstore',\append{\mkont'}{\mkont''}})}
           {\\ &\phantom{= \{}(\tpl{\mpoint,\mstore,\mkont}, \tpl{\mpoint',\mstore',\mkont'}) \in R,
            \\ &\phantom{= \{}\mkont'' \in \tails(\mkont, \mktab)} \\
\text{where } \tails(\kmt, \mktab) &= \set{\kmt} \\
              \tails(\krt{\mctx}, \mktab) &= \setbuild{\mkont' \in \unroll{\mktab}{\mkont}}{\mkont \in \mktab(\mctx)} \\
              \tails(\kcons{\mkframe}{\mkont}, \mktab) &= \tails(\mkont,\mktab)
\end{align*}

The $\mathit{unroll}$ relation interprets what are all the valid continuations that $\mktab$ encodes for a given continuation that contains $\krt{\mctx}$, defined inductively by the following rules:
\begin{mathpar}
  \inferrule{ }{\kmt \in \unroll{\mktab}{\kmt}} \quad
  \inferrule{\mkont \in \unroll{\mktab}{\mkont'}}{\mkframe:\mkont \in \unroll{\mktab}{\mkframe:\mkont'}} \\
  \inferrule{\mkont' \in \mktab(\mctx) \\
             \mkont \in \unroll{\mktab}{\mkont'}}
            {\mkont \in \unroll{\mktab}{\krt{\mctx}}}  
\end{mathpar}

For a pushdown analysis, we can simply take the least fixed-point of ${\mathcal F}$, but in general we can only say a witnessed reduction after $n$ steps in one case will in the other case witness the same reduction in some $m$ steps.
%
In the limit, the CESK reduction relation unfolds to the reification of the PDCFA system, and vice versa.
%
One small caveat is that state selection from $F$ must be fair: for all states $\mastate \in F$ there is some $m$ such that $\mastate$ not in $F$ component of ${\mathcal F}(\mastate_0)^m(\tpl{F,R,\mktab,\mmemo})$.
%
A queue representation of $F$ suffices to guarantee fairness.

\begin{theorem}[Correctness of Summarization]\label{thm:concrete-tabular}
  $\forall \mexpr$ closed, let $\mstate_0 = \tpl{\mexpr,\bot\bot,\kmt}$ in
  $\forall n \in \nat$:
  \begin{itemize}
  \item{if $(\mstate,\mstate') \in \reachrestrict(\mstate_0,\stepto_\mathit{CESK},n,\emptyset)$ then
      $\exists m \in \nat. (\mstate,\mstate') \in \reify({\mathcal F}(\mstate_0)^m(\emptyset,\emptyset,\bot,\bot))$}
  \item{
      if $(\mstate,\mstate') \in \reify({\mathcal F}(\mstate_0)^n(\emptyset,\emptyset,\bot,\bot))$ then
      $\exists m \in \nat. (\mstate,\mstate') \in \reachrestrict(\mstate_0,\stepto_\mathit{CESK},m,\emptyset)$}
  \end{itemize}
  given comparable allocation strategies and fair state selection.
\end{theorem}

We unfold a reduction relation from a starting state in the following way:
\begin{align*}
  \reachrestrict(\mstate_0, \stepto, 0, R) &= R \\
  \reachrestrict(\mstate_0, \stepto, n+1, R) &= \reachrestrict(\mstate_0, \stepto, n, R') \\
  \textit{where } R' &= \setbuild{(\mstate_0,\mstate')}{\mstate_0 \stepto \mstate'}
  \\ &\cup \setbuild{(\mstate,\mstate')}{(\_,\mstate) \in R, \mstate \stepto \mstate'}
\end{align*}

Allocation strategies $\alloc$ and $\alloc^*$ are comparable if they produce equal addresses regardless of the differences in the state representations:
\begin{mathpar}
  \inferrule{\mkont \in \unroll{\mktab}{\mkont'}}
            {\alloc(\tpl{\mpoint,\mstore,\mkont}) = \alloc^*(\tpl{\mpoint,\mstore,\mkont',\mktab,\mmemo})}
\end{mathpar}

We build toward a proof of correctness by giving invariant characterizations of the $\mktab$ and $\mmemo$ tables (\autoref{fig:inv}), which further depend on a ``context irrelevance'' lemma about the CESK machine.
%
Our development is partially verified in Coq, available with the Redex models in \texttt{model.v}; propositions marked with ${}^*$ are mechanically checked.

\begin{lemma}[Context irrelevance${}^*$]\label{lem:stack-irrelevance}
  For all CESK traces $\mtrace$ and continuations $\mkont$ such that $\hastail(\mtrace,\mkont)$,
  for any $\mkont'$, $\replacetail(\mtrace,\mkont,\mkont')$ is also a valid trace.
\end{lemma}
Where $\hastail$ (``has tail'') is
\begin{mathpar}
  \inferrule{ }{\hastail(\epsilon,\mkont)} \quad
  \inferrule{ }{\hastail(\tpl{\mpoint,\mstore,\append{\mkont'}{\mkont}},\mkont)} \quad
  \inferrule{\hastail(\mtrace\mstate,\mkont) \quad
             \mstate \stepto \mstate' \quad
             \hastail(\mstate',\mkont)}
            {\hastail(\mtrace\mstate\mstate',\mkont)}
\end{mathpar}
and $\replacetail$ (``replace tail'') is
\begin{align*}
  \replacetail(\tpl{\mpoint,\mstore,\append{\mkont'}{\mkont}},\mkont,\mkont'') &= \tpl{\mpoint,\mstore,\append{\mkont'}{\mkont''}} \\
  \replacetail(\epsilon,\mkont,\mkont'') &= \epsilon \\
  \replacetail(\mtrace\mstate,\mkont,\mkont') &= \replacetail(\mtrace,\mkont,\mkont')\replacetail(\mstate,\mkont,\mkont') \\
\end{align*}
Proof by induction on $\mtrace$ and cases on $\stepto$.

The first invariant is that the memo table only contains information about previously seen contexts; we need this to infer that there was at least one call leading to the memoized context such that we can use stack irrelevance to justify skipping to the memoized result.
%
Second, there is a path from the starting state of the current continuation to the current state.
%
Third, similar paths exist for all continuations stored in $\mktab$.
%
Finally, there is some continuation that stays in the tail of the trace of a memoization entry from starting context to final results (so we can interchange continuations with context irrelevance to justify memo table uses).
%
%The second, $\phi_{\reachable}$, states that all the calling contexts in the continuation table reach some unrolling of the current state.
%
%The final invariant, $\phi_{\memo}$, states that all paths starting at a function call either reach the memoized result, or if the path does not include a return, there is an extension that will.
%
%The portion of these paths between call and return hal the calling context's continuation in the tail to justify using stack-irrelevance.

\begin{figure}
  \centering
  \begin{align*}
    \startstate(\kmt) &= \tpl{\mexpr_\mathit{initial}, \bot, \bot, \kmt} \\
    \startstate(\krt{\mctx}) &= \tpl{\mctx, \kmt} \\    
    \startstate(\kcons{\mkframe}{\mkont}) &= \startstate(\mkont)
 \\[2pt]
    \inv(\tpl{\mpoint, \mstore, \mkont, \mktab, \mmemo}) &=
    \dom(\mmemo) \subseteq \dom(\mktab) \\ % domain containment
    % path to current kont
    &\wedge \startstate(\mkont) \stepto^* \tpl{\mpoint, \mstore,\append{\mkont}{\kmt}} \\
    % ktable meaning
    &\wedge \forall (\mpoint',\mstore') \in \dom(\mktab), \mkont''
    \in \mktab(\mpoint',\mstore'). \\
    &\quad\startstate(\mkont'') \stepto^* \tpl{\mpoint',\mstore',\append{\mkont''}{\kmt}} \\
    % memo table meaning
    &\forall(\mpoint',\mstore') \in \dom(\mmemo).
      \forall (\mval,\mstore'') \in \mmemo(\mpoint',\mstore').
      \\ & \quad
       \exists \mkont, \mtrace\equiv\tpl{\mpoint',\mstore',\mkont}\stepto^* \tpl{\mval,\mstore'',\mkont}.
         \hastail(\mtrace,\mkont)
  \end{align*}
  \caption{Table invariants}
\label{fig:inv}
\end{figure}
\begin{lemma}[Table invariants in CESK$\mathit{\Xi{}M}^*$]\label{lem:tab-inv}
  If $\inv(\mstate)$ and $\mstate \stepto \mstate'$ then $\inv(\mstate')$.
\end{lemma}
  By cases on $\stepto$.

As the later developments in this paper are of a similar character, we will describe correctness claims as a system reflecting a relation that has a bisimulation with the original semantics.