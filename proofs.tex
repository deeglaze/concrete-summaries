\section{In depth proofs}
Proof of \autoref{lem:irrelevance}
\begin{proof}
  By induction on $\mtrace$. Base cases trivially hold, and induction step follows from cases on $\mstate \stepto \mstate'$ and definition of $\replacetail$.
\end{proof}

\noindent{}Proof of \autoref{lem:tab-inv}
\begin{proof}
  By cases on $\stepto$.
  \begin{byCases}
% variable lookup
    \case{\tpl{(\svar\mvar, \menv), \mstore, \mkont, \mktab, \mmemo}
          \stepto
          \tpl{\mval,\mstore,\mkont, \mktab, \mmemo} \text{ if }\mval \in \mstore(\menv(\mvar))}{
     Same continuation and tables, so holds by $\inv(\mstate)$.}
% application
    \case{\tpl{(\sapp{\mexpri0}{\mexpri1}, \menv), \mstore, \mkont, \mktab, \mmemo}
          \stepto
          \tpl{(\mexpri0, \menv), \mstore, \kcons{\kar{\mexpri1,\menv}}{\mkont}, \mktab, \mmemo}}{
    Let $\mctx, \mkont^\circ, \mkont'$ be arbitrary.
    Let $\mkont^*$ be the witness for $\mstate$.
    The next witness is $\kcons{\kar{\mexpri1,\menv}}{\mkont^*}$, by definition of $\mathit{unroll}$.}
% argument evaluation
    \case{\tpl{\mval, \mstore, \kcons{\kar{\mexpr,\menv}}{\mkont}, \mktab, \mmemo}
          \stepto
          \tpl{(\mexpr, \menv), \mstore, \kcons{\kfn{\mval}}{\mkont}, \mktab, \mmemo}}{
    Let $\mctx, \mkont^\circ, \mkont'$ be arbitrary.
    Let $\kcons{\kar{\mexpr,\menv}}{\mkont^*}$ be the witness for $\mstate$.
    The next witness is $\kcons{\kfn{\mval}}{\mkont^*}$, by definition of $\mathit{unroll}$.}
% function call
    \case{\tpl{\mval,\mstore,\kcons{\kfn{\vclo{\slam{\mvar}{\mexpr}}{\menv}}}{\mkont},\mktab,\mmemo}
          \stepto
          \tpl{\mpoint, \mstore', \krt{\mpoint, \mstore'}, \mktab', \mmemo}}{
     where
     $\begin{array}{l}
       \mpoint = (\mexpr, \extm{\menv}{\mvar}{\maddr}) \\
       \mstore' = \joinone{\mstore}{\maddr}{\mval} \\
       \mktab' = \joinone{\mktab}{(\mpoint, \mstore')}{\mkont}
     \end{array}$ \\

    Let $\mctx, \mkont^\circ, \mkont'$ be arbitrary.
    Let $\kcons{\kfn{\mval}}{\mkont^*}$ be the witness for $\mstate$.
    The next witness is $\mkont^*$, by definition of $\mathit{unroll}$.}
% memo-lookup
    \case{\tpl{\mval,\mstore,\kcons{\kfn{\vclo{\slam{\mvar}{\mexpr}}{\menv}}}{\mkont},\mktab,\mmemo}
          \stepto
          \tpl{\mval_\mathit{result}, \mstore'', \mkont, \mktab', \mmemo}
          \text{ if } (\mval_\mathit{result},\mstore'') \in \mmemo(\mpoint,\mstore')}{
        same where clause as previous case.
     
    Let $\mctx, \mkont^\circ, \mkont'$ be arbitrary.
    By definition of $\inv$, and since $\tpl{\mpoint',\mstore',\mkont'} \stepto^* \tpl{\mval_{\mathit{result}}, \mstore'', \mkont'}$.
    For the memo entry to exist, there must have been at least one previous continuation in $\mktab(\mpoint',\mstore')$.
    We use $\inv$ with this continuation, combined with stack irrelevance to produce the trace involving $\mkont'$.
    }
% return
    \case{\tpl{\mval, \mstore, \krt{\mpoint,\mstore'}, \mktab, \mmemo}
          \stepto
          \tpl{\mval, \mstore, \mkont, \mktab, \joinone{\mmemo}{(\mpoint, \mstore')}{(\mval,\mstore)}}
          \text{ if } \mkont \in \mktab(\mpoint, \mstore')}{
    Let $\mctx, \mkont^\circ, \mkont'$ be arbitrary.
    $\mkont'$ is an acceptable witness for the first property.
    The path we construct to $\mstate'$ is sufficient for the second property.}
  \end{byCases}
\end{proof}

Proof of \autoref{lem:irrelevance} is by simple induction on traces.
% \noindent{}Proof of \autoref{thm:pdcfa-tabular}, which is more formally stated as (tabular $\pdstepto$, CESK $\stepto$): There is a well-founded equivalence bisimulation between $\pdstepto$ (quotiented to states satisfying $\inv$) and $\tred{\stepto}$.
% We define equivalence ($\simeq$) as
% \begin{mathpar}
%   \inferrule{\mkont \in \unroll{\mktab}{\mkont'} \\ \mkont \nequiv \kcons{\kfn{\mval}}{\mkont''}}
%             {\mtrace\tpl{\mpoint,\mstore,\mkont} \simeq \tpl{\mpoint,\mstore,\mkont,\mktab,\mmemo}} \\
% %
%   \inferrule{\mkont \in \unroll{\mktab}{\mkont'} \\
%              ((\mexpr,\menv[\mvar \mapsto \maddr]), \mstore\sqcup[\maddr \mapsto \set{\mval}]) \notin \dom(\mmemo)}
%             {\mtrace\tpl{\mval,\mstore,\kcons{\kfn{\slam{\mvar}{\mexpr}, \menv}{\mkont}}} \simeq \tpl{\mval,\mstore,\kcons{\kfn{\slam{\mvar}{\mexpr},\menv}}{\mkont'},\mktab,\mmemo}} \\
% %
%   \inferrule{\tpl{\mval', \mstore', \mkont} \notin \mtrace'' \\
%              \mkont \in \unroll{\mktab}{\mkont''} \\
%              ((\mexpr,\menv[\mvar \mapsto \maddr]), \mstore\sqcup[\maddr \mapsto \set{\mval}]) \in \dom(\mmemo)}
%             {\mtrace'\tpl{\mval, \mstore, \kcons{\kfn{\slam{\mvar}{\mexpr}, \menv}}{\mkont}}\mtrace''
%               \simeq \tpl{\mval, \mstore, \kcons{\kfn{\slam{\mvar}{\mexpr}, \menv}}{\mkont''}, \mktab, \mmemo}}
% \end{mathpar}
% This means that states which ``line up'' regardless of history are related, and that states that follow a function call that has not yet completed are related to the function call state where we already have the memoized results.
% %
% Each step in this case will get ``closer to'' the result to memoize, as given by the length of the path we get from the invariant on the memo table.
% \begin{proof}
%   Let $\mtrace,\mtrace',\mstate_\Xi$ be arbitrary such that $\mtrace \simeq \mstate_\Xi$ and $\mtrace \tred{\stepto} \mtrace'$.
%   By cases on $\mtrace \simeq \mstate_\Xi$:
%   \begin{byCases}
%     \case{\text{First rule}}{One step related by simple case analysis.}
%     \case{\text{Second rule}}{One step related by definitions of $\mathit{unroll}$, $\stepto$ and $\pdstepto$.}
%     \case{\text{Third rule}}{
%     By cases on $\mtrace\mstate \tred{\stepto} \mtrace\mstate\mstate'$:
%     \begin{byCases}
%       \case{\mtrace\tpl{(\svar\mvar, \menv), \mstore, \mkont} \stepto
%         \mtrace\mstate\tpl{\mval,\mstore,\mkont}}{...}
%       % application
%       \case{\mtrace\tpl{(\sapp{\mexpri0}{\mexpri1}, \menv), \mstore,
%           \mkont} \stepto \mtrace\mstate\tpl{(\mexpri0, \menv),
%           \mstore, \kcons{\kar{\mexpri1,\menv}}{\mkont}}}{...}
%       % argument evaluation
%       \case{\mtrace\tpl{\mval, \mstore,
%           \kcons{\kar{\mexpr,\menv}}{\mkont}} \stepto
%         \mtrace\mstate\tpl{(\mexpr, \menv), \mstore,
%           \kcons{\kfn{\mval}}{\mkont}}}{...}
%       % function call
%       \case{\mtrace\tpl{\mval,\mstore,\kcons{\kfn{\vclo{\slam{\mvar}{\mexpr}}{\menv}}}{\mkont}}
%         \stepto \mtrace\mstate\tpl{(\mexpr,
%           \extm{\menv}{\mvar}{\maddr}),
%           \joinone{\mstore}{\maddr}{\mval}, \mkont}}{...}
%     \end{byCases}}
%   \end{byCases}
%   By \autoref{lem:tab-inv}
  
%   Soundness is a corollary of \autoref{lem:tab-inv}.
% \end{proof}

% \noindent{}Proof of \autoref{thm:global}.
% \begin{proof}
% %% TODO
%   The invariants of the memo and k-tables are independent of the states that carry them, so lifting them to a global position does not affect their information.
% \end{proof}

% \noindent{}Proof of \autoref{thm:refinement}.
% \begin{proof}
%   todo
% \end{proof}

% \noindent{}Proof of \autoref{thm:cfa2}.
% \begin{proof}
%   todo
% \end{proof}

% \noindent{}Proof of \autoref{thm:concrete-sr}.
% \begin{proof}
%   todo
% \end{proof}

% \noindent{}Proof of \autoref{thm:sound-sr}
% \begin{proof}
%   todo
% \end{proof}