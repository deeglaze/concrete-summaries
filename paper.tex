\documentclass{llncs}

\usepackage{amsmath,stmaryrd,natbib}
\usepackage[usenames,dvipsnames]{color}

\newcommand{\Scribtexttt}[1]{{\texttt{#1}}}
\newcommand{\SColorize}[2]{\color{#1}{#2}}
\newcommand{\inColor}[2]{{\Scribtexttt{\SColorize{#1}{#2}}}}
\definecolor{PaleBlue}{rgb}{0.90,0.90,1.0}
\newcommand{\rackett}[1]{\inColor{black}{#1}}
% Metavariables from defined spaces
\newcommand{\mexpr}{e}
\newcommand{\mexpri}[1]{e_{#1}}
\newcommand{\mvar}{x}
\newcommand{\mval}{v}
\newcommand{\mvalpre}{v_\mathit{pre}}
\newcommand{\mvalpost}{v_\mathit{post}}
\newcommand{\mtag}{v_{\mathit{t}}}
\newcommand{\mhandler}{v_\mathit{h}}
\newcommand{\maddr}{a}
\newcommand{\menv}{\rho}
\newcommand{\mstore}{\sigma}
\newcommand{\mkont}{\kappa}
\newcommand{\mskont}{\hat{\kappa}}
\newcommand{\mmkontacc}{C_\mathit{acc}}
\newcommand{\mmkont}{C}
\newcommand{\mstate}{\varsigma}
\newcommand{\mastate}{\hat\varsigma}

\newcommand{\mprim}{\mathit{prim}}

\newcommand{\mktab}{\Xi}
\newcommand{\mmemo}{M}
\newcommand{\mpoint}{p}
\newcommand{\mctx}{\mathit{ctx}}
\newcommand{\msctx}{\mathit{sctx}}
\newcommand{\mframe}{\xi}
\newcommand{\mkframe}{\phi}
\newcommand{\mlab}{\ell}
\newcommand{\mtrace}{\pi}

\newcommand{\callcc}{\rackett{call/cc}}
\newcommand{\callcomp}{\rackett{call/comp}}
\newcommand{\abort}{\rackett{abort}}
\newcommand{\dynamicwind}{\rackett{dynamic-wind}}

% Metavariables for Spaces
\newcommand{\Var}{\mathit{Var}}
\newcommand{\Addr}{\mathit{Addr}}
\newcommand{\Expr}{\mathit{Expr}}
\newcommand{\Store}{\mathit{Store}}
\newcommand{\Kont}{\mathit{Kont}}
\newcommand{\SKont}{\widehat{\mathit{Kont}}}
\newcommand{\MKont}{\mathit{MKont}}
\newcommand{\Env}{\mathit{Env}}
\newcommand{\State}{\mathit{State}}
\newcommand{\System}{\mathit{System}}
\newcommand{\Value}{\mathit{Value}}
\newcommand{\Prims}{\mathit{Prims}}

\newcommand{\Context}{\mathit{Context}}
\newcommand{\SContext}{\mathit{StorableContext}}

\newcommand{\Point}{\mathit{Point}}
\newcommand{\Frame}{\mathit{Frame}}

\newcommand{\KTab}{\mathit{KTable}}
\newcommand{\Memo}{\mathit{Memo}}

% Other metavariables (metafunction names, etc)
\newcommand{\alloc}{\mathit{alloc}}
\newcommand{\bind}{\mathit{bind}}
\newcommand{\lookup}{{\mathcal L}}
\newcommand{\wn}{\mathit{wn}}
\newcommand{\dom}{\text{dom}}
\newcommand{\onep}{\mathit{One?}}
\newcommand{\reverse}{\mathit{reverse}}
\newcommand{\revapp}{\mathit{rev\text{-}append}}
\newcommand{\finalize}{\mathit{finalize}}
\newcommand{\captureupto}{\mathit{capture\text{-}upto}}
\newcommand{\aborttargets}{\mathit{abort\text{-}targets}}

\newcommand{\returns}{\mathit{returns}}
\newcommand{\memoize}{\mathit{memoize}}
\newcommand{\approximate}{\mathit{approximate}}
\newcommand{\touches}{{\mathcal T}}
\newcommand{\touchesf}{{\mathcal T}_{\mkframe}}
\newcommand{\reaches}{{\mathcal R}}
\newcommand{\konts}{\mathit{konts}}

\newcommand{\inv}{\mathit{inv}}
\newcommand{\reachable}{\mathit{reach}}
\newcommand{\memo}{\mathit{memo}}
\newcommand{\unroll}[2]{\mathit{unroll}_{#1}(#2)}
\newcommand{\hastail}{\mathit{hastail}}
\newcommand{\replacetail}{\mathit{replacetail}}

% Constructors
\newcommand{\svar}[2][{}]{#2^{#1}}
\newcommand{\sapp}[3][{}]{(#2\ #3)#1}
\newcommand{\slam}[3][{}]{\lambda#1 #2. #3}
\newcommand{\sprim}[1]{#1}
\newcommand{\sreset}[1]{(\texttt{reset}\ #1)}
\newcommand{\sshift}[2]{(\texttt{shift}\ #1. #2)}
\newcommand{\vcont}[1]{\mathbf{cont}(#1)}
\newcommand{\vcomp}[1]{\mathbf{comp}(#1)}
\newcommand{\vclo}[2]{(#1, #2)}
\newcommand{\kar}[2][{}]{\mathbf{ar}#1(#2)}
\newcommand{\kpush}[1]{\mathbf{push}(#1)}
\newcommand{\klet}[1]{\mathbf{lt}(#1)}
\newcommand{\kfn}[2][{}]{\mathbf{fn}#1(#2)}
\newcommand{\rt}{\mathbf{rt}}
\newcommand{\krt}[1]{\rt(#1)}
\newcommand{\kmt}{\mathbf{mt}}

\newcommand{\kprompt}[1]{\#(#1)}
\newcommand{\dw}{\mathbf{dw}}
\newcommand{\kdw}[1]{\dw(#1)}
\newcommand{\abrt}{\mathbf{abrt}}
\newcommand{\kabrt}[1]{\abrt(#1)}
\newcommand{\kev}[1]{\mathbf{ev}(#1)}

\newcommand{\mextag}[4]{(\#^{#2}_{#3} #1\ #4)}
\newcommand{\kmatch}[4]{(?^{#2}_{#3} #1\ #4)}
\newcommand{\mkapp}[2]{#1\circ #2}

% Notations
\newcommand{\lfp}{\mathbf{lfp}}
\newcommand{\tpl}[1]{\langle #1 \rangle}
\newcommand{\vect}[1]{\langle #1 \rangle}
\newcommand{\runpost}[1]{\langle #1 \rangle_{\mathit{post}}}
\newcommand{\runhandler}[1]{\langle #1 \rangle_{\mathit{handler}}}
\newcommand{\cstate}[1]{\langle #1 \rangle_{\mathit{call}}}
\newcommand{\set}[1]{\{ #1 \}}
\newcommand{\setbuild}[2]{\{ #1\ :\ #2\}}
\newcommand{\sa}[1]{\widehat{#1}}

\newcommand{\many}[1]{\overline{#1}}
\newcommand{\kcons}[2]{#1{\tt :} #2}

\newcommand{\snglm}[2]{[#1 \mapsto \set{#2}]}
\newcommand{\extm}[3]{#1[#2 \mapsto #3]}
\newcommand{\joinm}[3]{#1\sqcup[#2 \mapsto #3]}
\newcommand{\joinone}[3]{#1\sqcup[#2 \mapsto \set{#3}]}

\newcommand{\append}[2]{#1 {\tt ++} #2}

%% Relations
\newcommand{\stepto}{\longmapsto}
\newcommand{\astepto}{\mathrel{\widehat{\longmapsto}}}

\title{Concrete Semantics for Pushdown Analysis:\\The Essence of Summarization}
\author{J. Ian Johnson and David Van Horn}
\institute{Northeastern University \\
           \email{\{ianj,dvanhorn\}@ccs.neu.edu}}

\usepackage{hyperref}
\begin{document}
\maketitle

% Outline:
% Introduction.
% High level
% PDCFA
% - Concrete
% - Abstract
% CFA2
% - Concrete
% - Abstract
% WCM machine
% - Concrete
% - Abstract
% Related Work
% Future Work
% Conclusion

\section{What to expect}
This paper lays out a common framework to talk about pushdown analysis
for higher-order languages. In particular, we extract the ``essence''
of the existing techniques into a quality of the machine semantics
that specify a language's meaning, much in the same way as
~\citet{dvanhorn:VanHorn2012Systematic} (a method we will refer to as
AAM: abstracting abstract machines). Once the machine semantics is in
this form, simple pointwise abstraction leads to the summarization
algorithms that we see in the literature. In effect, we give a
\emph{concrete} semantics to pushdown analysis.

Summarization algorithms need not be restricted to languages with
well-bracketed calls and returns. We can adopt the technique for
higher precision in the common case but still handle difficult cases
such as first-class control. This was shown for the
\rackett{call-with-current-continuation} (a.k.a. \rackett{call/cc})
operator in ~\citet{ianjohnson:Vardoulakis2011Pushdown}. This
impressive work illuminated the fact that we can harness the enhanced
technology of pushdown analyses in non-pushdown models of
computation. Doing this sacrifices call/return matching in the general
case, but in practice the precision is much better than the
alternative regular\footnote{Regular as in regular language.} model
that, say, AAM would provide. A downside of the work is that it was an
algorithmic change to the already complicated CFA2 - there was no
recipe for how to do this for one's favorite control operator. Instead
of deriving a ``pushdown'' analysis for a language with
\rackett{call/cc}, we will show a new analysis for a language with the
all the control operators in
\citet{ianjohnson:Flatt:2007:ADC:1291151.1291178} (call it the PLT
machine) in order to demonstrate the applicability of this viewpoint
even in the context of complex control operators.

The remaining sections of the paper are
\begin{itemize}
\item{section \ref{sec:pdcfa}: we derive a cousin of PDCFA}
\item{section \ref{sec:cfa2}: we make additions to the previous semantics to get a direct-style CFA2 without first-class control}
\item{section \ref{sec:plt}: we give a novel analysis of the PLT machine, a core calculus of Racket's control operations.}
\end{itemize}

\section{Deriving PDCFA}
\label{sec:pdcfa}
PDCFA is a simpler analysis than CFA2, and thus enjoys a polynomial
complexity in the monovariant case. Its final ${\mathcal O}(n^6)$ form
unfortunately is not a complete abstraction of an unbounded stack
model. In this section, we show how to derive a ${\mathcal O}(n^9)$
complete abstraction\footnote{Interestingly, the available
  \emph{implementations} of PDCFA incorporate the fix and are
  themselves ${\mathcal O}(n^9)$.}. These bounds are absolute worst
case that would never ever happen in practice. The complete
abstraction restricts value flow to fewer spurious paths, often
leading to a faster analysis than a typical 0CFA.

The magic of the method is in keeping a table of (local) continuations
for each function $\times$ store pair. This closely mirrors the
technique of store-allocating continuations used in AAM. Instead of
deferring to the table (store) for the remainder of a continuation for
each frame, we only have indirections at function call
boundaries. Note that since AAM is about finitizing the state space,
this step itself fits within the AAM method, since continuations
within functions truncated at call boundaries are bounded by the
nesting depth of those functions, thereby making the continuation
space finite. In the monovariant case, the number of continuations is
still linear in the size of the program.

The tail of a continuation in the context of a function will contain
the stackless context in which the function was called, in order to
link up with the proper callsite(s). The context includes the function
(or unique label of the function), the environment, and the store. If
doing a flow analysis that takes advantage of the pushdown model, the
context would also have a lattice element.

Once we finitize the address space of the store, this computes a
characteristic finite state machine that has only reachable states of
the pushdown system the unbounded stack model embodies.

A ``summary edge'' that is in the literature, in terms of pushdown
systems, is an edge from the source of a push edge to the target of
the matching pop edge. ``Matching'' here means there is a path through
machine edges that don't change the stack, or through summary
edges. There is an analogy to something more operational: summary
edges embody memoization. Instead of following an entire path through
a call to return a value, we simply jump from the call to the return
(the result of the call). We are very conservative with how much of
the context is necessary to memoize, so we use the entire heap. This
conservativeness isn't useful for skipping any work in computing the
flow analysis (so much so that we don't even bother consulting the
memo table in the semantics), but it is useful for interpreting the
results to talk about the behavior of the body of a function without
detours to other functions. We will return to this analogy and what it
means in the presence of first-class control in section \ref{sec:plt}.

\begin{figure}
  \begin{align*}
    \mexpr \in \Expr &= \svar{\mvar} \mid \sapp{\mexpr}{\mexpr} \mid \slam{\mvar}{\mexpr} \\
    \mstate \in \State &= (\Expr \times \Env) \times \Store \times \Kont \\
    \mval \in \Value &::= (\slam{\mvar}{\mexpr},\ \menv) \\
    \mkont \in \Kont &::= \kmt \mid \kar{\mexpr,\menv,\mkont} \mid \kfn{\mval,\mkont} \\
    \menv \in \Env &= \Var \to \Addr \\
    \mstore \in \Store &= \Addr \to \wp(\Value) \\
    \Var &\text{ an infinite set} \\
    \Addr &\text{ an infinite set}
  \end{align*}
  \caption{The CESK semantic spaces}
\label{fig:cesk-spaces}
\end{figure}

\begin{figure}
  \centering
  $\mstate \stepto \mstate' \text{ where } \maddr = \alloc(\mstate)$ \\
  \begin{tabular}{r|l}
    \hline
% variable lookup
    $\tpl{(\svar\mvar, \menv), \mstore, \mkont}$
    &
    $\tpl{\mval,\mstore,\mkont}$ if $\mval \in \mstore(\menv(\mvar))$
    \\
% application
    $\tpl{(\sapp{\mexpri0}{\mexpri1}, \menv), \mstore, \mkont}$
    &
    $\tpl{(\mexpri0, \menv), \mstore, \kar{\mexpri,\menv,\mkont}}$
    \\
% argument evaluation
    $\tpl{\mval, \mstore, \kar{\mexpr,\menv,\mkont}}$
    &
    $\tpl{(\mexpr, \menv), \mstore, \kfn{\mval, \mkont}}$
    \\
% function call
    $\tpl{\mval,\mstore,\kfn{\slam{\mvar}{\mexpr}, \mkont}}$
    &
    $\tpl{(\mexpr, \extm{\menv}{\mvar}{\maddr}), \joinone{\mstore}{\maddr}{\mval}, \mkont}$
  \end{tabular}
  \caption{The CESK machine}
  \label{fig:base-semantics}
\end{figure}

\begin{figure}
  \begin{align*}
    \mstate \in \State &= (\Expr \times \Env) \times \Store \times \Kont \times \KTab \times \Memo \\
    \mkont \in \Kont &::= \kmt \mid \kar{\mexpr,\menv,\mkont} \mid \kfn{\mval,\mkont} \mid \krt{(\mexpr,\menv),\mstore}\\
    \KTab &= (\Expr \times \Env)\times \Store \to \wp(\Kont) \\
    \Memo &= (\Expr \times \Env)\times \Store \to \wp(\Value)
  \end{align*}
  \caption{Extended semantic spaces}
\label{fig:cesk-ext-spaces}
\end{figure}


\begin{figure}
  \centering
  $\mstate \stepto \mstate' \text{ where } \maddr = \alloc(\mstate)$ \\
  \begin{tabular}{r|l}
    \hline
% variable lookup
    $\tpl{(\svar\mvar, \menv), \mstore, \mkont, \mktab, \mmemo}$
    &
    $\tpl{\mval,\mstore,\mkont, \mktab, \mmemo}$ if $\mval \in \mstore(\menv(\mvar))$
    \\
% application
    $\tpl{(\sapp{\mexpri0}{\mexpri1}, \menv), \mstore, \mkont, \mktab, \mmemo}$
    &
    $\tpl{(\mexpri0, \menv), \mstore, \kar{\mexpri1,\menv,\mkont}, \mktab, \mmemo}$
    \\
% argument evaluation
    $\tpl{\mval, \mstore, \kar{\mexpr,\menv,\mkont}, \mktab, \mmemo}$
    &
    $\tpl{(\mexpr, \menv), \mstore, \kfn{\mval, \mkont}, \mktab, \mmemo}$
    \\
% function call
    $\tpl{\mval,\mstore,\kfn{\slam{\mvar}{\mexpr}, \mkont},\mktab,\mmemo}$
    & % actually do call,
    $\tpl{\mpoint,
          \mstore',
          \krt{\mpoint, \mstore'},
          \joinone{\mktab}{(\mpoint, \mstore')}{\mkont},
          \mmemo}$
    \\ % or, lookup in table
%    & $\tpl{\mval, \mstore', \mkont,\mktab,\mmemo}$ if $\mval \in \mmemo(\mpoint, \mstore')$ \\    
    where & $\mpoint = (\mexpr, \extm{\menv}{\mvar}{\maddr})$ \\
          & $\mstore' = \joinone{\mstore}{\maddr}{\mval}$
    \\
% return
    $\tpl{\mval, \mstore, \krt{\mpoint,\mstore'}, \mktab, \mmemo}$
    &
    $\tpl{\mval, \mstore, \mkont, \mktab, \joinone{\mmemo}{(\mpoint, \mstore')}{\mval}}$
    if $\mkont \in \mktab(\mpoint, \mstore')$
  \end{tabular}
  \caption{The summarizing tabular stack machine}
  \label{fig:summary-semantics}
\end{figure}

To turn this semantics into PDCFA's algorithm for constructing a Dyck
state graph, we apply a widening operator ${\mathcal F}$ to make $\mstore$, $\mkont$,
and $\mmemo$ shared amongst all states. Further techniques for a
performant implementation can be found in
\citet{ianjohnson:oaam:2013}.

\begin{align*}
  \mastate \in \sa{State} &= \Expr \times \Env \times \Kont \\
  \System &= \wp(\sa{State} \times \Store) \times \wp(\sa{State}) \times \Store \times \KTab \times \Memo \\
  {\mathcal F} &: \System \to \System \\
  {\mathcal F}(S, F, \mstore, \mktab, \mmemo) &= (S \cup S', F', \mstore', \mktab', \mmemo') \\
  \text{where } I &= \setbuild{\mstate'}{\mastate \in F, \wn(\mastate,\mstore,\mktab,\mmemo) \stepto \mstate'} \\
                \mstore' &= \bigsqcup\setbuild{\mstore'}{\wn(\_, \mstore',\_,\_) \in I} \\
                \mktab' &=  \bigsqcup\setbuild{\mktab'}{\wn(\_, \_, \mktab', \_) \in I} \\
                \mmemo' &=  \bigsqcup\setbuild{\mmemo'}{\wn(\_, \_, \_, \mmemo') \in I} \\
                S' &= \setbuild{(\mastate',\mstore')}{\wn(\mastate', \_, \_, \_) \in I} \\
                F' &= \setbuild{\mastate'}{(\mastate',\_) \in S', (\mastate',\mstore') \notin S} \\
                \wn(\tpl{(\mexpr,\menv),\mkont},\mstore,\mktab,\mmemo) &= \tpl{(\mexpr,\menv),\mstore,\mkont,\mktab,\mmemo}
\end{align*}

\section{Deriving CFA2}
\label{sec:cfa2}

CFA2 is the first published analysis of a higher-order programming
language that could properly match calls and returns. Vardoulakis and
Shivers had a clear goal of harnessing the extra information a
pushdown model provides to produce a high-precision analysis that
works well in practice. This resulted in more than just the
call/return matching of the previous section, which is why we are
showing the two separately. There are two extra features of the
semantics:
\begin{enumerate}
\item{stack allocation for some bindings}
\item{strong updates on stack frames for resolved nondeterminism}
\end{enumerate}
The first of these is an addition to the stackless context. There is a
conservative pre-analysis that checks locally whether a binding will
never escape, and classifies references as able to use the stack frame
or not. Their criteria for a binding never escaping is that it is
never referenced in a function that is not its binder. This can be
extended in a language with more linguistic features; see Kranz's
thesis about register-allocatable bindings in the Orbit Scheme
compiler \citep{ianjohnson:kranz:thesis:1988}. A stackable reference
is one that appears in the body of the binding function, by which we
mean not within the body of nested function. They use the information
that a binding never escapes to not bother allocating it in the
heap. This has the advantage of not changing the heap, and thus leads
to less propagation. The addition of these stack frames makes the
analysis exponential in theory, though in practice they have proved to
decrease running time in most cases.

The second of these is to ameliorate a problem they call ``fake
rebinding.'' That is, since bindings in the abstract represent several
values, we don't want to reference a variable \texttt{x} in two
different places and have it resolve to two different values. In AAM,
a variable reference non-deterministically steps to all possible
values associated with that variable. Here we want to say that once
\texttt{x} is considered to stand for value $v$, then all subsequent
references of \texttt{x} should be $v$. If they aren't, it looks as if
\texttt{x} was rebound; it hasn't, and thus it is a ``fake
rebinding.'' CFA2 does not step to all values on variable reference,
but instead carries all its values around in superposition until they
need to be observed at, say, a function call. We give a simplified
semantics that is more along the AAM style, but CFA2's approach can
easily be recovered from it.

We show only the significantly modified rules of the semantics in
figure \ref{fig:frame-semantics}. The other rules simply carry along
the extra $\mframe$ component untouched.
\begin{figure}
  \centering
  $\mstate \stepto \mstate' \text{ where } \maddr = \alloc(\mstate)$ \\
  \begin{tabular}{r|l}
    \hline
% variable lookup
    $\tpl{(\svar[\mlab]{\mvar}, \menv), \mstore, \mframe, \mkont}$
    &
    $\tpl{\mval,\mstore,\mframe',\mkont}$ if $(\mframe', \mval) \in \lookup(\mstore,\mframe,\menv,\mvar,\mlab)$
    \\
% % application
%     $\tpl{(\sapp{\mexpri0}{\mexpri1}, \menv), \mstore, \mframe, \mkont}$
%     &
%     $\tpl{(\mexpri0, \menv), \mstore, \mframe, \kar{\mexpri1,\menv,\mkont}}$
%     \\
% % argument evaluation
%     $\tpl{\mval, \mstore, \mframe, \kar{\mexpr,\menv,\mkont}}$
%     &
%     $\tpl{(\mexpr, \menv), \mstore, \mframe, \kfn{\mval, \mkont}}$
%     \\
% function call
    $\tpl{\mval,\mstore,\mframe,\kfn{\slam{\mvar}{\mexpr}, \mkont}}$
    &
    $\tpl{(\mexpr, \extm{\menv}{\mvar}{\maddr}), \mstore', \mframe', \mkont}$
    \\ where & $(\mstore',\mframe') = \bind(\mstore,\mframe,\maddr,\mvar,\mval)$
  \end{tabular}
  \caption{The CES$\xi$K machine}
  \label{fig:frame-semantics}
\end{figure}

\begin{align*}
  \bind(\mstore,\mframe,\maddr,\mvar,\mval) &=
   \left\{\begin{array}{ll}
            (\mstore, \snglm{\maddr}{\mval}) & \text{if } \mvar \text{ never escapes} \\
            (\joinone{\mstore}{\maddr}{\mval}, \snglm{\maddr}{\mval}) & \text{otherwise } \\
          \end{array}\right. \\
  \lookup(\mstore,\mframe,\menv,\mvar,\mlab) &=
    \left\{\begin{array}{ll}
          \setbuild{(\extm{\mframe}{\menv(\mvar)}{\set{\mval}}, \mval)}
                   {\mval \in \mframe(\menv(\mvar))}
                   & \text{if } \mlab \text{ non-escaping} \\
          \setbuild{(\mframe,\mval)}{\mval \in \mstore(\menv(\mvar))} & \text{otherwise}
           \end{array}\right.
\end{align*}

\section{Analysis of the PLT machine}
\label{sec:plt}

There is contention among programming language researchers whether
\rackett{call/cc} should be a language primitive, since it captures
the entire stack, leading to space leaks
\citep{ianjohnson:kiselyov:against-callcc}. Alternative control
operators have been proposed that delimit how much of the stack to
capture, such as Felleisen's $\%$ (read ``prompt'') and capture
operator ${\mathcal F}$ (read ``control'')
\citep{ianjohnson:felleisen:control:1988}, or Danvy and Filinski's
\texttt{reset} (equivalent to $\%$) and \texttt{shift}
\citep{ianjohnson:danvy:filinski:delim:1990}. However, the stacks
captured by these operators always extend the stack when invoked,
rather than replace it like those captured with
\rackett{call/cc}. Continuations that have this extension behavior are
called ``composable continuations.'' Stack replacement is simple to
model in a regular analysis using the AAM approach, and Vardoulakis
and Shivers showed it can be done in a pushdown approach (although it
breaks the pushdown model). Stack extension, however, poses a new
challenge for pushdown analysis, since one application of a composable
continuation means pushing an unbounded amount of frames onto the
stack. Vardoulakis' and Shivers' approach does not immediately apply
in this situation, since their technique drops all knowledge of the
stack at a continuation's invocation site; extension, however, must
preserve it.

Composable and non-composable continuations both have their uses, as
it has been shown the two are not co-expressive when it comes to space
complexity ~\citet{ianjohnson:dybvig:control:2007}. There is still
merit in delimiting the stack in either case, as is argued by
\citet{ianjohnson:Flatt:2007:ADC:1291151.1291178}. Their paper gives a
semantics for the coexistence of composable and non-composable
continuations in the presence of first-class prompts,
\rackett{dynamic-wind} (protects entry/exit of a continuation with
pre- and post-processors), and continuation marks. We will show how to
apply pushdown analysis techniques in the presence of all this
complexity.

The way we have been splitting continuations at function calls has
similarities to the meta-continuation approach to modeling delimited
control ~\citep{ianjohnson:Biernacki2006274}. We could view each
function call as inserting a prompt, and returns as aborting to the
nearest prompt. This view contends with tail calls, but we can
identify tail calls easily - any call with a prompt as its
continuation is a tail call. Tail calls are important in language
implementations for space complexity reasons
\citep{ianjohnson:clinger:tail-calls:1998}, but in an analysis, these
concerns are less important. The repeated popping of prompts inserted
by what otherwise were tail-calls is synonymous with CFA2's
``transitive summaries.''

We first give a CESK semantics of the PLT machine with the allocation
procedure as a parameter in order to first discuss the complexities
any computable abstract semantics will have. We then present the key
ideas to making it a table-based semantics, and how we can abstract it
to a computable approximation.

\subsection{PLT's CESK semantics}

The treatment of prompts as first-class values and
\rackett{dynamic-wind}'s interaction with non-composable continuations
requires equality tests of dynamically generated values. Because of
this, once we relax the freshness condition for $\alloc$, we need a
semantics that can handle the possibility that two values aren't just
equal or disequal - they \emph{may} be equal. This complication is
separate from the pushdown approach we are presenting, so we give it a
separate presentation in this section to subsequently amend with pushdown
techniques.

...TODO...

\subsection{Tabular PLT semantics}

Since continuations can now appear in the store, and the store is in
the $\krt{}$ continuation, we introduced a circularity in the data
represented in our machine - this is precisely what the AAM approach
avoids in order to have a finite model. There is currently no known
model of computation that allows stack capture and
replacement/composition that has decidable reachability, so we choose
to approximate in this case. We represent only a bounded amount of a
continuation, after which we say that any continuation matching that
prefix is acceptable. The amount to represent is a tunable parameter
that we simply set to be one of the local continuations we already
have in our model.

\section{Related Work}

AAM, PDCFA, CFA2, CFA2+, Reps

\section{Conclusion}

\bibliographystyle{chicago}
\bibliography{bibliography}

\end{document}