\documentclass{llncs}

\usepackage{amsmath,stmaryrd,natbib}
\usepackage[usenames,dvipsnames]{color}

\newcommand{\Scribtexttt}[1]{{\texttt{#1}}}
\newcommand{\SColorize}[2]{\color{#1}{#2}}
\newcommand{\inColor}[2]{{\Scribtexttt{\SColorize{#1}{#2}}}}
\definecolor{PaleBlue}{rgb}{0.90,0.90,1.0}
\newcommand{\rackett}[1]{\inColor{black}{#1}}

\title{Concrete Pushdown Summarization}
\author{J. Ian Johnson}
\institute{Northeastern University \\
           \email{ianj@ccs.neu.edu}}

\begin{document}
\maketitle

% Outline:
% Introduction.
% High level
% PDCFA
% - Concrete
% - Abstract
% CFA2
% - Concrete
% - Abstract
% WCM machine
% - Concrete
% - Abstract
% Related Work
% Future Work
% Conclusion

\section{What to expect}
This paper lays out a common framework to talk about pushdown analysis
for higher-order languages. In particular, we extract the ``essence''
of the existing techniques into a quality of the machine semantics
that specify a language's meaning, much in the same way as
~\citet{dvanhorn:VanHorn2012Systematic}. Once the machine semantics
is in this form, simple pointwise abstraction leads to the
summarization algorithms that we see in the literature. In effect, we
give a \emph{concrete} semantics to pushdown analysis.

Summarization algorithms need not be restricted to languages with
well-bracketed calls and returns. We can adopt the technique for
higher precision in the common case but still handle difficult cases
such as first-class control. This was shown for the
\rackett{call-with-current-continuation} (a.k.a. \rackett{call/cc})
operator in ~\citet{ianjohnson:Vardoulakis2011Pushdown}. This
impressive work illuminated the fact that we can harness the enhanced
technology of pushdown analyses in non-pushdown models of
computation. Doing this sacrifices call/return matching in the general
case, but in practice the precision is much better than the
alternative regular model. A downside of the work is that it was an
algorithmic change to the already complicated CFA2 - there was no
recipe for how to do this for one's favorite control operator. Instead
of deriving a ``pushdown'' analysis for a language with
\rackett{call/cc}, we will show a new analysis for a language with the
all the control operators in \citet{ianjohnson:Flatt:2007:ADC:1291151.1291178} (call it the
PLT machine) in order to demonstrate the applicability of this
viewpoint even in the context of complex control operators.

The remaining sections of the paper are
\begin{itemize}
\item{section \ref{sec:pdcfa}: we derive a cousin of PDCFA}
\item{section \ref{sec:cfa2}: we make additions to the previous semantics to get a direct-style CFA2 without first-class control}
\item{section \ref{sec:plt}: we give a novel analysis of the PLT machine, a core calculus of Racket's control operations.}
\end{itemize}

\section{Deriving PDCFA}
\label{sec:pdcfa}

\section{Deriving CFA2}
\label{sec:cfa2}

\section{Analysis of the PLT machine}
\label{sec:plt}

\bibliographystyle{chicago}
\bibliography{bibliography}

\end{document}